\documentclass[11pt]{beamer}
\title{Using Dynamic Analysis to Generate Disjunctive Invariants}
\usepackage{verbatim}
\usepackage{amsmath}
\usepackage{amsthm}
\usepackage{graphics}
\usepackage{color}
\usepackage{stmaryrd}

\newtheorem{proposition}{Proposition}


\date{\today}


\begin{document}
\maketitle
\begin{frame}\frametitle{Introduction}
\begin{itemize}
\item Invariants: defect detection, program verification and program repair.

\item Find the invariants: static or dynamic, and their pros and cons.

\item Conjunctive, polynomial and convex invariants $\Rightarrow$ Disjunctive program properties

\end{itemize}

\end{frame}
\begin{frame}\frametitle{Disjunctive Program}
\begin{example}
\[\texttt{if }(p)\{a = 1;\}\texttt{ else }\{a=2;\}\]
\end{example}
Neither $a = 1$ nor $a = 2$ is an invariant, but $(p \wedge a = 1)\vee (\neg p \wedge a = 2)$ is an invariant.
\end{frame}
\begin{frame}\frametitle{Overview}
\begin{itemize}
\item Existing invariant inference algorithm.


\item Max-Plus invariants and a way to infer them.


\item Using $k$-induction to verifying candidate invariants.


\item Experiment results.

\end{itemize}
\end{frame}
\end{document}