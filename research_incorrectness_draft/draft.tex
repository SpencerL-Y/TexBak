\documentclass[a4paper,12pt]{article}
\usepackage{geometry}
\usepackage{verbatim}
\usepackage{amsmath}
\usepackage{amsthm}
\usepackage{amssymb}
\usepackage{mathtools}
\usepackage{listings}
\usepackage{graphicx}
\usepackage{color}
\usepackage{stmaryrd}
\usepackage{multirow}

\title{Discussion on Octagon}
\newtheorem{definition}{Definition}
\newtheorem{example}{Example}
\newcommand{\cona}{\mathcal{A}}
\newcommand{\conb}{\mathcal{B}}
\newcommand{\conc}{\mathcal{C}}
\newcommand{\cond}{\mathcal{D}}
\geometry{a4paper, scale = 0.8}
\begin{document}
\maketitle

\section{Draft}
\paragraph{Synthesizing an Octagon Predicate $p$}

\begin{definition}[Octagon]
Given a set of variables $X$ where all variables in the set belongs to a numerical set $\mathbb{I}$, which can be $\mathbb{Z}, \mathbb{R}$ or $\mathbb{Q}$. We call \emph{octagonal constraint} any constraint of the form $\pm x_i \pm x_j \ge c$ where $c\in\mathbb{I}$ and $x_i,x_j\in X$. An \emph{octagon} is the set of points that satisfies the conjunction of all octagonal constraints. 


\end{definition}
Assume the program we consider is affine linear. From the definition of incorrectness logic and the iteration rule, our target is to synthesize a predicate $p(\mathbf{x},n)$ for a loop program where the update of the loop body can be expressed as $\mathbf{x}' = M\mathbf{x}$, s.t. 
\[\models \forall \mathbf{x}.n. (p(\mathbf{x},n+1) \implies \exists \mathbf{y}.\mathbf{x} = M\mathbf{y} \wedge p(\mathbf{y}, n))\]
After the elimination of the existential quantifier we get:
\[\models \forall \mathbf{x}.n. (p(\mathbf{x}, n+1) \implies p(k_0(\mathbf{x} - \mathbf{c}) + k_1\mathbf{v}_i, n))\]



\begin{example}

We first consider the simplest example where $X = \{x, n\}$, i.e. $\mathbf{x}$ only contains one variable. We assume the update of the program is $x' = ax + b$. The octagon is equivalently given by the form:


\begin{align*}
x + y \ge& \cona_{x,y}\\
x - y \ge& \conb_{x,y}\\
-x + y \ge& \conc_{x,y}\\
-x - y \ge& \cond_{x,y}\\
\end{align*}
where $x,y\in X$.

For this example, then the constraint system $S_1$ of $p(\mathbf{x}, n+1)$ can be given as:

\begin{align*}
2x&& &\ge \cona_{x,x}\\
&&0 &\ge \conb_{x,x}\\
&&0 &\ge \conc_{x,x}\\
-2x&& & \ge \cond_{x,x}\\
x &+ n& +1&\ge \cona_{x,n}\\
x &- n& +1&\ge \conb_{x,n}\\
-x &+ n& +1&\ge \conc_{x,n}\\
-x &- n&+1 &\ge \cond_{x,n}\\
&2n & +2&\ge \cona_{n,n}\\
&& 0&\ge \conb_{n,n}\\
&& 0&\ge \conc_{n,n}\\
&-2n& -2&\ge\cond_{n,n}\\
\end{align*}
Similarly, from the fact that $\mathbf{y} = [y] = [\frac{1}{a} x - \frac{b}{a}]$, we can also derive a system $S_2$ for $p(\mathbf{y}, n)$:

\begin{align*}
\frac{2}{a}x&& -\frac{2b}{a} &\ge \cona_{x,x}\\
&&0 &\ge \conb_{x,x}\\
&&0 &\ge \conc_{x,x}\\
-\frac{2}{a}x&& \frac{2b}{a} &\ge \cond_{x,x}\\
\frac{1}{a}x &+ n& -\frac{b}{a}&\ge \cona_{x,n}\\
\frac{1}{a}x &- n& -\frac{b}{a}&\ge \conb_{x,n}\\
-\frac{1}{a}x &+ n& +\frac{b}{a}&\ge \conc_{x,n}\\
-\frac{1}{a}x &- n& +\frac{b}{a}&\ge \cond_{x,n}\\
&2n &&\ge \cona_{n,n}\\
&& 0&\ge \conb_{n,n}\\
&& 0&\ge \conc_{n,n}\\
&-2n&&\ge\cond_{n,n}\\
\end{align*}

Target of the synthesis is to synthesize the unknown parameter $\{\cona, \conb, \conc, \cond\}$ s.t. $\models \forall x.n. (S_1\implies S_2)$.

According to the method in previous SAS'06:
\end{example}


\end{document}