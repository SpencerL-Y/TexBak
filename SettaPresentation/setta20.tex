\documentclass[11pt]{beamer}
\usetheme{Rochester}
\usecolortheme{beaver}
\title{Computing Linear Arithmetic Representation of Reachability Relation  of One-counter Automata}
\usepackage{beamerfoils}
\usepackage{tikz}
\usepackage{pgf}
\usepackage{verbatim}
\usepackage{amsmath}
\usepackage{amsthm}
\usepackage{listings}
\usepackage{graphics}
\usepackage{color}
\usepackage{stmaryrd}
\usepackage{multicol}
\usepackage{tabularx}
\usepackage{listings}
\usepackage{tikz}
\usepackage{stmaryrd}\usefonttheme[onlymath]{serif}

\usetikzlibrary{automata,shapes,arrows,patterns,calc,positioning}
\tikzset{every picture/.style={->, >=stealth', shorten >=0pt, auto, initial text={}}}
\tikzstyle{flowchartelement} = [draw, inner sep=3pt, align=center]
\tikzstyle{inputoutput} = [flowchartelement, trapezium, trapezium right angle=-70pt, trapezium left angle=70pt]
\tikzstyle{test} = [flowchartelement, diamond]
\tikzstyle{code} = [flowchartelement, minimum width=20mm, rectangle]
\tikzstyle{connector} = [draw, circle, fill=black, inner sep=1.25pt]
\newtheorem{proposition}{Proposition}
\author{
    Authors: \textbf{Xie Li}, Taolue Chen, Zhilin Wu and Mingji Xia
}
\date{SETTA 2020: Guangzhou, China, November 24-28, 2020}

\titlegraphic{\includegraphics[width=3cm]{iscaslogo.png}\hspace*{0.75cm}~%
\includegraphics[width = 2cm]{birkbeck.jpg}
\hspace*{0.75cm}~%
\includegraphics[width = 3cm]{ucaslogo.png}
}


\begin{document}
\maketitle
\begin{frame}\frametitle{Overview}
\begin{itemize}
\item Introduction to One-counter Automata(OCA) and its Reachability Relation.
\pause
\item Computing the Reachability Relation of OCA.
\pause
\item Tool \textsc{OCAReach} and Experimental Results.

\end{itemize}
\end{frame}


\begin{frame}\frametitle{What is One-counter Automata(OCA)}
\begin{itemize}

\item DFA with a \textbf{counter} $c$ where $c$ is a \textbf{non-negative} integer.
\pause
\item Transitions: $q\stackrel{\texttt{Op}}{\rightarrow} q'$ where $\texttt{Op}\in \{\texttt{add}(+1), \texttt{add}(-1), \texttt{zero}\}$
\end{itemize}
\pause
\begin{example}[OCA]
\begin{center}
$\langle c\rangle$
\includegraphics[scale=1]{ocaexample.eps}
\end{center}


\end{example}

\end{frame}
\begin{frame}\frametitle{Semantic of OCA}

Semantic of OCA: A transition system.
\begin{itemize}

\item Configuration: $(q,c)$.

\item Transitions of configurations corresponds to the transitions in the OCA.

\end{itemize}
\[(q_1, c_1) \rightarrow_\mathcal{A} (q_2,c_2)\]

if $q_1 \stackrel{\texttt{add}(+1)}{\longrightarrow} q_2$ in the OCA and $c_1 + 1 = c2$ and $\texttt{add}(-1)$ vice versa, or

if $q_1 \stackrel{\texttt{zero}}{\longrightarrow} q_2$ and $c_1 = c_2 = 0$.


$c_1, c_2 \ge 0$
\end{frame}

\begin{frame}\frametitle{Reachability Problem of OCA}

\textbf{Reachability Problem of OCA: } whether $(q_s, c_s)\rightarrow^*_{\mathcal{A}} (q_t, c_t)$
\pause
\begin{example}\frametitle{Reachability Problem} 
\begin{center}
\includegraphics[scale=1]{reachexample2.eps}

\end{center}
Due to the non-negative requirement,

$(q_1, 1)$ is not reachable from $(q_0, 0)$




\end{example}
\end{frame}

\begin{frame}\frametitle{Reachability Relation}

\textbf{Reachability Problem of OCA: } whether $(q_s, c_s)\rightarrow^*_{\mathcal{A}} (q_t, c_t)$

Instead of using concrete values $c_s$ and $ c_t$, we use variables $x_s$ and $x_t$ for the reachability relation.
\pause

\begin{definition}[Reachability Relation of OCA]
A reachability relation of an OCA $\mathcal{A}$ from state $q_s$ to $q_t$ is a set $R_{\mathcal{A}, q_s, q_t} \subseteq \mathbb{N} \times \mathbb{N}$. 

\[\forall (c_s, c_t)\in R_{\mathcal{A}, q_s, q_t}. (q_s, c_s)\rightarrow^*_{\mathcal{A}} (q_t, c_t)\]
\end{definition}
\pause
Goal: use a Presburger Arithmetic (PA) formula $\phi(x_s, x_t)_{\mathcal{A}, q_s, q_t}$ to represent this relation.
\end{frame}


\begin{frame}\frametitle{Weighted Directed Graph,  Path and Flow}
\begin{itemize}

\item An OCA can be regarded as a weighted directed graph $G_\mathcal{A} = (V,E,\eta)$.
\pause
\item \textbf{Path}: a sequence of vertices $v_0\cdot v_1 \cdots v_k$ where $(v_i, v_{i+1}) \in E$.
\pause
\begin{itemize}
\item Drop of path
\end{itemize}
\pause
\item \textbf{Flow}: a function $f: E\rightarrow \mathbb{N}$.
\end{itemize}

\begin{example}
\begin{center}
\includegraphics[scale=1]{reachexample2.eps}
\includegraphics[scale=1]{wg.eps}
\begin{itemize}

\item path: $v_0\cdot v_1\cdot v_1\cdot v_1$
\item drop: $-1$
\end{itemize}
\end{center}


\end{example}
\end{frame}

\begin{frame}\frametitle{Path Flow and Support}


\begin{itemize}


\item Support: edge-induced subgraph of flow.

\item  $s$-$t$ \textbf{path flow}: the flow corresponds to a path.
\begin{itemize}
\item Requirements of flow of each vertex.
\item Connectivity of the support.
\end{itemize}

\end{itemize}
\pause

\[\phi^{pf}_{G_\mathcal{A}, s,t}(x_s, x_t)_{f_{e\in E}}:= weight(f) = x_t - x_s \wedge\]
\[\textbf{if }(s = t) \textbf{ then foreach } v\in V: \texttt{num}(\text{in-flows})_v = \texttt{num}(\text{out-flows})_v\]
\[\textbf{else} \textbf{ foreach } v\in V-\{s,t\}: (\texttt{num}(\text{in-flows})_v = \texttt{num}(\text{out-flows})_v \wedge\]
\[\texttt{num}(\text{in-flows})_s = \texttt{num}(
\text{out-flows})_s - 1 \wedge \]
\[\texttt{num}(\text{in-flows})_t = \texttt{num}(\text{out-flows})_t + 1 ) \]

\end{frame}

\begin{frame}\frametitle{Example of Path Flow}

\begin{example}
\begin{multicols}{2}

\includegraphics[scale=1]{wg2.eps}
\begin{itemize}

\item Support: 

\includegraphics[scale=1]{support.eps}

\item Path: $v_0\cdot v_1\cdot v_1 \cdot v_1$

\item Pathflow: $f(v_0,v_0) = 0$ 

$f(v_0,v_1) = 1$

$ f(v_1,v_1) = 2$

\item Weight: $weight(f) = \Sigma_{e\in E} f(e)\cdot \eta(e)$
\end{itemize}
\end{multicols}

\end{example}
\end{frame}

\begin{frame}\frametitle{The Difficulty of Computing the Reachability Relation}
\begin{center}
\large
\textbf{NON-NEGATIVE}
\end{center}\pause
If we do not require the non-negative of counter.

\begin{itemize}
\item Existence of a path flow.
\end{itemize}

\pause
\[\phi_{G_{\mathcal{A}}, s,t}(x_s, x_t)_{f_{e\in E}}\]
\textbf{Non-negative} implies the constraint:
everywhere along the path, the counter need to be non-negative.
\end{frame}

\begin{frame}\frametitle{Certificates of the Reachability}

Use \textbf{path flow} as certificate of OCA reachability problem.
\pause
\begin{itemize}
\begin{multicols}{2}
\item Type-1 Certificate: 
\begin{itemize}
\item Flow is a path flow.
\item No positive cycle.
\item Path flow has edge decompositions (which implies \textbf{non-negative}).

\end{itemize}
\begin{center}

\includegraphics[scale=1]{type1-sim.eps}
\end{center}
\end{multicols}

\pause


\item Type-3 Certificate:
\begin{center}
\includegraphics[scale=1]{type2.eps}
\end{center}
\pause
\item Type-2 Certificate: Dual of type-1 certificate at the end.

\end{itemize}


\end{frame}


\begin{frame}\frametitle{Edge Decomposition}

\begin{definition}[Edge Decomposition]
Edge decomposition of a path flow $f$ is 
\begin{itemize}
\pause
\item $f = \Sigma_{i\in |E|} f_{i} + f_{e_i}$ where $f_{i}$ is also a path flow, $e_i\in E$.
\pause
\item $e_i$ does not appear in support of $f_{j}$ where $j > i$.
\pause
\item $weight(f_{i}) + weight(e_i) \ge 0$ for all $i$.
\end{itemize}

\end{definition}
\pause

This definition implies the non-negative requirement of Type-1 certificate.
\begin{center}
\includegraphics[scale=1.2]{edgeDecom.eps}
\end{center}
\end{frame}

\begin{frame}\frametitle{Decidability of Reachability of OCA}
\begin{theorem}[Haase]
The reachability problem of OCA can be solved iff we can find a certificate that is of the form

\[(\text{Type-1})^{n_1}(\text{Type-3})^{n_3}(\text{Type-2})^{n_2}\]

where $n_i\in \{0,1\}$
\end{theorem}
\end{frame}

\iffalse

\begin{frame}\frametitle{Reachability Relation of OCA}

\textbf{Reachability Relation of OCA}: 
\[\phi_{{\mathcal{A}}, q_s, q_t}(x_s, x_t)\]


$\phi_{{\mathcal{A}}, q_s, q_t}(c_s, c_t)$ is true iff $(q_s, c_s)\rightarrow^*_{\mathcal{A}} (q_t, c_t)$.
\end{frame}

\fi
	

\begin{frame}\frametitle{Use PA Formula to Represent Type-3 Reachability Relation}
\[\phi_{G_{\mathcal{A}}, s, t}(x_s, x_t) =\exists (f_e)_{e\in E}. \phi^{T1RC} \vee \phi^{T2RC} \vee \phi^{T3RC} \vee \cdots \]
\textbf{Type-3 certificate:}
\[\phi^{T3RC}_{G_\mathcal{A}, s, t}(x_s, x_t)_{ (f_{e,3})_{e\in E}} \]
\begin{itemize}
\item Positive Cycle at $q_s$.
\item Existence of a $q_s$-$q_t$ path flow.
\item Negative Cycle at $q_t$.
\end{itemize}
\end{frame}


\begin{frame}\frametitle{Use PA Formula to Represent Type-1 Reachability Relation}
\[\phi^{T1RC}_{G_\mathcal{A}, s, t}(x_s, x_t)_{(f_{e,1})_{e\in E}} := \exists (idx_e, sum_e)_{e\in E}\]
\pause
\[\phi^{pf}_{G_\mathcal{A}, s, t}(x_s, x_t)_{f_{e\in E}} \wedge\pause\phi^{APC}(f_{e\in E}) \wedge\pause \]
\[\phi^{EDC}((f_e, idx_e, sum_e)_{e\in E}), (f_{e,e'})_{e,e'\in E})\]


\end{frame}
\begin{frame}\frametitle{\textsc{OCAReach}: Experimental Evaluation}
Implemented in Java and utilzing \textsc{Z3} solver for formula manipulation.

\textbf{INPUT:} file describing the OCA.

\textbf{OUTPUT:} a PA formula $\phi$ representing reachability relation.
\begin{itemize}
\item Experiment on handcrafted cases.
\begin{center}
\includegraphics[scale=0.4]{expoca.png}
\end{center}
\item On random cases.
\end{itemize}
\end{frame}

\begin{frame}\frametitle{Contributions and Future Work}
Contributions:
\begin{itemize}
\item 
Some work to make computation of reachability relation possible.
\item 

We built the gap between the theory and implementation by the tool \textsc{OCAReach}.

\end{itemize}
Future work:

\begin{itemize}

\item Optimize our tool to improve the efficiency.

\item More and larger cases and find benchmarks for experiment.

\item Other topics about one-counter automata.
 
\end{itemize}
\end{frame}

\begin{frame}
\begin{center}
\textbf{Thanks! \& Questions?}
\end{center}
\end{frame}

\end{document}