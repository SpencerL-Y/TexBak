\documentclass[11pt]{beamer}
\title{One-Counter Automata and its Reachability Problem} 
\usepackage{verbatim}
\usepackage{amsmath}
\usepackage{amsthm}
\usepackage{listings}
\usepackage{graphics}
\usepackage{color}
\usepackage{stmaryrd}\usefonttheme[onlymath]{serif}

\newtheorem{proposition}{Proposition}
\author{Przemyslaw Daca et al.}
\date{\today}


\begin{document}
\maketitle
\begin{frame}\frametitle{Overview }
\begin{itemize}
\item Preliminaries: Counter Automata and Pushdown Automata and Turing Machine.


\item Reachability Problem.

\item Some Interesting Reductions.


\item Algorithm.

           
\end{itemize}
\end{frame}

\begin{frame}\frametitle{Reference}
[1] Haase et al. On the Complexity of Model Checking Counter Automata. Phd thesis.

[2] Marvin L. Minsky. Recursive unsolvability of post’s problem of ”tag” and othertopics in theory of turing machines.The Annals of Mathematics, 74(3):437–455,1961.

[3] Xie Li, Taolue Chen, Zhilin Wu and Mingji Xia. Computing Linear Arithmetic Representation for Reachability Relation of One-counter Automata

[4] Wikipedia..
\end{frame}

\begin{frame}\frametitle{Counter Automata}
\begin{definition}[k-Counter Automata]

Let $k\in \mathbb{N}_{>0}$ and $\texttt{Op} = \{\texttt{add}_i(z)\mid i\in[k], z\in \mathbb{Z}\} \cup \{\texttt{zero}_i\}$. A $k$-counter automaton is a tuple $\mathcal{A} = (Q, q_0, F, \Delta, \epsilon)$ where $\epsilon: \Delta \rightarrow \texttt{Op}$ is a additional transition labelling function.


\end{definition}

\begin{example}{1-Counter Automata}



\end{example}\hspace{5000pt}
\end{frame}

\begin{frame}\frametitle{Pushdown Automata}
\begin{definition}[PDA]
A pushdown automaton is a tuple $M = (Q, \Sigma, \Gamma, \delta, q_0, Z, F)$.
\begin{itemize}
\item $\delta$ is a finite subset of $Q\times (\Sigma \cup \{\epsilon\}) \times \Gamma \times Q \times \Gamma^*$.
\item $\Sigma, \Gamma$ are the tape alphabet and stack alphabet respectively.
\item $Z$ is the initial symbol of stack representing the bottom of the stack.
\end{itemize}

\end{definition}

\end{frame}
\begin{frame}\frametitle{One-Counter Automata and Pushdown Automata}

\textbf{One-counter automata is a special case of pushdown automata.}

$\Gamma = \{Z, g\}$ and the stack of PDA can be written as 

\[Zg^n, n\in \mathbb{N}\]

which can be regarded as a non-negative counter.



\end{frame}


\begin{frame}\frametitle{Two-Counter Automata is Turing Equivalent}
Proof sketch:
\begin{enumerate}
\item A Turing machine can be simulated by two stacks.
\item A stac kcan be simulated by two counters. Where one counter is used to storing the binary number represented by the stack and the other one used for scratchpad for update. 

\item Four counters can be simulated by two counters. Four virtual counter $a,b,c,d$ can be encoded as a G\"odel number $2^a3^b5^c7^d$ by one real counter and the comparablyother counter is used as scratchpad.
\end{enumerate}
\end{frame}

\begin{frame}\frametitle{Undecidable problem of TM}
\begin{itemize}
\item Acceptance problem. $\langle TM, \omega\rangle$.
\item Reachability problem. $TM, c_{init}, c_{f}$.

\end{itemize}
More generally, Rice's theorem formally state what problem is decidable about turing machine.

\begin{theorem}{Rice's Theorem}
If $P$ is a non-trivial property, and the language holding the property, $L_P$ is recognized by a turing machine $M$, then $L_P = \{\langle M\rangle\mid L(M) \in P\}$ is undecidable.
\end{theorem}


Hence, some important basic problems are all undecidable for counter automata.
\end{frame}


\begin{frame}\frametitle{Several Ways to Retain the Decidability}
\begin{itemize}
\item Restrict to 1-counter automata.

\item Structural restriction: flatness (no nested cycles).

\item Reversal Boundness.
\end{itemize}
\end{frame}


\begin{frame}\frametitle{Reachability Problem of OCA}

\begin{definition}[Configuration of OCA]
Given an OCA $\mathcal{A} = (Q, q_0, F, \Delta, \epsilon)$, we use a pair $(q, c)$ represent the configuration of $\mathcal{A}$.

\end{definition}

The semantic of an OCA can be regarded as a labelled transition system where the states are the configurations of $\mathcal{A}$ and the transitions are induced from $\Delta$ and $\epsilon$.


\textbf{REACHABILITY PROBLEM:}

Given an oca $\mathcal{A}$ and two configurations $(s, c_s), (t, c_t)$, whether we can find a feasible run of transition system $T_{\mathcal{A}}$ such that $(s,c_s)\rightarrow_{\mathcal{A}}^*(t,c_t)$.


\end{frame}


\begin{frame}\frametitle{Complexity of OCAReach Problem}

\begin{theorem}[Haase's Phd]
Reachability problem in one-counter automata is NP-complete.
\end{theorem}

\end{frame}

\begin{frame}\frametitle{Algorithm}
Why the reachability of OCA is hard?

The idea of the algorithm.

Our work.
\end{frame}

\end{document}