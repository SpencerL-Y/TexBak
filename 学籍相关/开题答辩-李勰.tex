
\documentclass[11pt]{beamer}
%aspectratio=1610
\usepackage{ctex}
\usepackage{CJKutf8}
\usepackage{xcolor}
\usepackage{multicol}
\usepackage{mathtools,array}
\usepackage[T1]{fontenc}
\usepackage{zi4}
\usepackage[font={scriptsize,bf}]{caption}
% \usepackage{subcaption}
\usepackage{graphics}
\usepackage{tikz}
\usepackage{fontawesome5}
\usepackage{mathpartir}

\newcommand{\naturals}{\mathbb{N}}
\newcommand{\reals}{\mathbb{R}}

\newcommand{\Dist}[1]{\mathcal{D}(#1)}
\newcommand{\expectation}{\mathbb{E}}

\newcommand{\states}{S}
\newcommand{\actions}{A}
\newcommand{\observables}{O}
\newcommand{\trans}{T}
\newcommand{\obs}{Z}
\newcommand{\reward}{R}
\newcommand{\discount}{\gamma}

\newcommand{\beliefs}{\mathcal{B}}
\newcommand{\beliefUpdate}{\tau}

\newcommand{\policy}{\pi}

\newcommand{\diff}[1]{\mathop{}\!\mathrm{d}#1}
\renewcommand{\figurename}{Figure}
\renewcommand{\refname}{Reference}

\AtBeginDocument{
  \catcode`_=12
  \begingroup\lccode`~=`_
  \lowercase{\endgroup\let~}\sb
  \mathcode`_="8000
}

% \usetheme{Madrid}
% % \usetheme{default}
% \setbeamertemplate{caption}[numbered]
% \setbeamerfont{title}{size=\large}
\mode<presentation>
{
  \usetheme{Darmstadt}      % or try Darmstadt, Madrid, Warsaw, ...
  \usecolortheme{default} % or try albatross, beaver, crane, ...
  \usefonttheme[onlymath]{serif}  % or try serif, structurebold, ...
  \setbeamertemplate{navigation symbols}{}
  \setbeamertemplate{caption}[numbered]
  \setbeamertemplate{footline}[frame number] 
} 

\usepackage{listings}
\lstdefinestyle{heaplang}{
    language=Caml,
    basicstyle=\footnotesize\ttfamily,
    keywordstyle=\color{blue},
    commentstyle=\color{red},
    escapeinside={<@}{@>},
    morekeywords={new_chan, fork, recv, send, swap, ref}
}
\lstdefinestyle{clang}{
    language=Caml,
    basicstyle=\footnotesize\ttfamily,
    keywordstyle=\color{blue},
    commentstyle=\color{red},
    escapeinside={<@}{@>},
}
\lstset{style=heaplang}

\usepackage{natbib}

\newcommand{\buchi}{B\"uchi }

\definecolor{goldenpoppy}{rgb}{0.99, 0.76, 0.0}
\definecolor{goldenyellow}{rgb}{1.0, 0.87, 0.0}
\definecolor{green2}{rgb}{0.1,0.7,0.3} 
\newcommand{\gcheck}{{\color{green2}\faCheckCircle[regular] }}
\newcommand{\rcross}{{\color{red} \faTimesCircle[regular]} }
\newcommand{\rflag}{{\color{red} \faFlag}}
% \usepackage{algorithm,amsmath}
% \usepackage[noend]{algpseudocode}

\newcommand{\zlstinline}{\let\par\endgraf\lstinline}
\newcommand{\comments}[1]{{\color{red}#1}}
\title{硕士开题答辩\\基于分离逻辑的内存安全分析与验证}
\date{2021年6月25日}
\author{学生:李勰 \qquad 导师:张立军}
\begin{document}
\maketitle

\begin{frame}\frametitle{提纲}
\begin{itemize}
\item 选题的背景及意义
\item 本题目的研究内容
\item 已有科研基础及研究计划
\end{itemize}
\end{frame}
\begin{frame}\frametitle{背景和意义}
\begin{itemize}
\item 什么是内存安全问题

\begin{itemize}
\item 程序中的内存操作:\texttt{malloc, free, load, store...}
\item 
常见的内存安全问题:内存泄漏、空指针解引用、访问越界等..
\item 可能出现的场景:嵌入式开发、网络编程、并发编程等
\end{itemize}


\item 内存安全的重要性

\end{itemize}
\end{frame}

\begin{frame}\frametitle{本题目的研究内容}
\begin{center}
基于分离逻辑的内存安全分析与验证
\end{center}
\pause

\begin{definition}[霍尔三元组]
霍尔三元组的语法形式:
\[\{P\}C\{Q\}\]
其中$P,Q$分别是用逻辑公式表示的前置条件和后置条件,$C$是执行的程序。
\end{definition}

\pause
这里$C$是含有内存操作的程序,程序的状态可以表示为

\[\texttt{States} = \texttt{Store}\times \texttt{Heap}\]
\[\texttt{Store}: Var \rightarrow Val, \quad \texttt{Heap}: Loc \rightarrow_+ (Loc \cup Val)\]


\end{frame}

\begin{frame}\frametitle{分离逻辑}
\begin{definition}[符号堆]
符号堆(Symbolic Heap)的语法形式:
\[\Pi \mid \Sigma \]
$\Pi$为纯公式$\Sigma$为空间公式。
\end{definition}
\begin{example}
\[ y = x + 3 \wedge x > 0 \mid \texttt{blk}(x, y)* y\mapsto 2\]
\[ i = 0\wedge j = 0 \mid \texttt{emp}\]
\[\texttt{true}\mid \texttt{blk}(a,b) * \texttt{blk}(c,d)\]
\end{example}
\end{frame}

\begin{frame}\frametitle{本题目研究内容}
基于已有的利用分离逻辑公式的符号执行思想,对关心的内存安全性质进行归约验证。
\begin{example}
\texttt{main}函数中的代码片段$C$,内存泄漏

\[\{\texttt{true}\mid \texttt{emp}\} C \{\texttt{true}\mid \texttt{emp}\}\]
\end{example}
需要的工作:
\begin{itemize}
\item 给出符合程序语义的分离逻辑推导规则
\item 根据性质,定位需要进行验证的位置,并生成验证条件
\end{itemize}
\end{frame}

\begin{frame}\frametitle{其他问题}
\begin{itemize}
\item 循环的处理
\begin{itemize}
\item 循环展开
\item 合成循环不变式 - 抽象解释 - 抽象域
\end{itemize}
\item 指针算术
\item 不确定的变量初始值
\end{itemize}
\end{frame}

\begin{frame}

\end{frame}

\end{document}