\documentclass[aspectratio=1610, 13pt]{beamer}
\usepackage{verbatim}
\usepackage{amsmath}
\usepackage{amsthm}
\usepackage{multicol}
\usepackage{graphics}
\usepackage{color}
\usepackage{stmaryrd}\usefonttheme[onlymath]{serif}

\usepackage{listings}
\usepackage{color}
\renewcommand\lstlistingname{Quelltext} % Change language of section name
\lstset{ % General setup for the package
    language=C,
    basicstyle=\small\sffamily,
    numbers=left,
     numberstyle=\tiny,
    tabsize=4,
    columns=fixed,
    showstringspaces=false,
    showtabs=false,
    keepspaces,
    commentstyle=\color{red},
    keywordstyle=\color{blue}
}


\title{A Local Shape Analysis based on Separation Logic}
\date{\today}
\author{Authors: Dino Distefano, Peter W. O'Heran and Hongseok Yang\\Presentor: Xie Li}
\begin{document}
\maketitle
\begin{frame}{Introduction}
    \begin{center}
        A \textbf{Local} \textbf{Shape Analysis} based on \textbf{Separation Logic}
    \end{center}
    
\begin{itemize}
    \item What is \textbf{Shape Analysis}?
    \item What is \textbf{Separation Logic}?
    \item How the analysis works?
    \item What does \textbf{Local} means?

\end{itemize}
\end{frame}



\begin{frame}{Shape Analysis}
\begin{itemize}
    \item Questions in heap content: NULL-pointers, May-Alias, Must-Alias,  Reachability, Disjointness, \textbf{Shape}.
    
    \item \textbf{Shapes} characterize data structures: singly linked list, linked list with cycle, doubly linked list, a binary tree...
    
    \item According to \footnote{Reinhard Wilheim et al. Shape Analysis. CC 2000.}, shape analysis computes for each point in the program:
    
    A finite, conservative representation of the heap-allocated data structures that could arise when a path to this program point executed.
\end{itemize}
    
\end{frame}

\begin{frame}[fragile]{Shape Analysis: Example}
\begin{multicols}{2}
\begin{example}
    \begin{lstlisting}
    List reverse(List x){
        List y, t;
        y = NULL;
        while(x != NULL){
            t = y;
            y = x;
            x = x->n;
            y->n = NULL;
            y->n = t;
        }
        return y;
    }
    \end{lstlisting}
\end{example}
\emph{Execution state}: cells, connectivity and values of pointer variables.
\begin{center}
    \includegraphics[scale=0.3]{shape_list.png}
\end{center}
\end{multicols}

    
\end{frame}

\begin{frame}{Shape Analysis: Example}
    Shape graph: 
\begin{center}
    \includegraphics[scale=0.4]{shape_graph.png}
\end{center}
Compute for each program point a set of shape graphs as a description of the superset of execution states at the point.

\end{frame}

\begin{frame}{Idea of the Paper}
    
    
    
    
    \begin{itemize}
        \item Problem of old shape analysis: updating of a abstract location may affect properties for other cells. (Frame rule)
        \item Locality between different separation conjunctions.
        \item Existing symbolic execution method for separation logic\footnote{Josh Berdine, Cristiano Calcagno and Peter W. O’Hearn. Symbolic execution with separation logic.}.
    \end{itemize}
\end{frame}

\begin{frame}{General Semantic Setting}
    Working with complete lattice: $D$. 
    
    $D$ is constructed from $\mathcal{P}(S')$, where $S' = S \cup \{\top\}$. $\top$ is a special element cprresponds to memory fault.
    
    semantic of a command $\llbracket c\rrbracket: D \rightarrow D$.
    
    \textbf{Semantic for key commands:}
    \begin{center}
        \includegraphics[scale=0.4]{semantic.png}
    \end{center}
    where $\texttt{filter}(b): D \rightarrow D$.
    
    \textbf{Execution semantic:}
    
    $p\Longrightarrow\subseteq S\times(S\cup \{\top\})$
    
    The execution semantic on the powerset of $S'$ is a function $\bar{p}: \mathcal{P}(S') \rightarrow \mathcal{P}(S')$: 
    \[\bar{p} X = \{\sigma'\mid \exists \sigma \in X. (\sigma, p \Longrightarrow \sigma') \vee (\sigma = \sigma' = \top)\}\]
\end{frame}

\begin{frame}{Separation Logic: Concrete and Symbolic Heaps}
    \begin{center}
        \includegraphics[scale=0.4]{conc_sema.png}
    \end{center}
    \begin{definition}[Symbolic Heap]
    A \emph{symbolic heap} $\Pi\mid \Sigma$ consists of a finite set $\Pi$ of \textbf{equalities} between \textbf{expressions} and a finite set $\Sigma$ of heap predicates, where
    \begin{itemize}
        \item Expressions $E,F$ are variables $x$, primed variables $x'$ or \texttt{nil}.
        \item Heap predicates includes $E\mapsto F$, $\texttt{ls}(E,F)$ and \texttt{junk}.
    \end{itemize}
    \end{definition}
    The meaning of symbolic heap corresponds to a formula:
    \begin{center}
        \includegraphics[scale=0.4]{sl_form.png}
    \end{center}
    We use $\mathcal{SH}$ to denote the set of consistent symbolic heaps.
\end{frame}

\begin{frame}{Separation Logic: Concrete and Symbolic Heaps}
Semantic of symbolic heap:
    \begin{example}
    \begin{itemize}
        \item $s,h\models E\mapsto F$: 
        
        $s_0(x) = 1, s_0(y) = 10$ and $h_0(1) = 10$. Then $s_0,h_0\models x\mapsto y$.
        \item $s,h\models \texttt{ls}(E,F)$: intuitively, this means we have a path from $E$ to $F$.
        \begin{itemize}
            \item $s_0(x) = 1, s_0(y) = 10$ and $h_0(1) = 10$. Then $s_0,h_0\models \texttt{ls}(x,y)$
            \item $s_1(x) = 1, s_1(y) = 2, s_1(z) = 3, s_1(w) = 4$ and $h_1(1) = 2, h_1(2)=3, h_1(3) = 4$. Then $s_1, h_1\models \texttt{ls}(x,w)$. Or $s_1, h_1\models x\mapsto y * \texttt{ls}(y,w)$.
        \end{itemize}
        
        
        \item $s,h\models \texttt{junk}$ iff $h\ne \emptyset$
    \end{itemize}
    Or graphically:
    \end{example}
\end{frame}

\begin{frame}{Questions of the Analysis}
The analysis requires us to answer some questions algorithmically:
    \begin{itemize}
        \item $\Pi\vdash E = F$
        
        Check by consider the equivalent classes.
        
        
        
        \item $\Pi\mid \Sigma \vdash \texttt{false}$
        \begin{itemize}
            \item $\Pi\vdash E = \texttt{nil}$ and $\texttt{allocated}(\Sigma, E)$
            \item $\Pi\vdash E = F$ and $\texttt{ls}(E,F)\in \Sigma$
            \item $\Pi\vdash E = F$ and the two distinct predicates whose lhs are $E,F$.
        \end{itemize}
        \item $\Pi\mid \Sigma \vdash E\ne F $ when $\texttt{Vars}'(E, F)=\emptyset$
        
        Add $E = F$ to the pure part and utilize analysis for \texttt{false}.
        
        \item $\Pi\mid \Sigma \vdash \texttt{Allocated}(E)$ when $ \texttt{Vars}'(E)=\emptyset$
    \end{itemize}
    All of the above analysis can be done syntactically.
\end{frame}

\begin{frame}{Questions of the Analysis: Example}
\begin{example}
    \begin{itemize}
        \item $\Pi\vdash E = F$
        
        Let $\Pi$ be $\{x = x', x' = y', y' = y, z = \texttt{nil}\}$. There are two equivalent classes of the variables: $\{x, x', y, y'\}$ and $\{z, \texttt{nil}\}$. From this information we have $\Pi\vdash x = y$ and $\Pi\vdash z = nil$.
        
        
        
        \item $\Pi\mid \Sigma \vdash \texttt{false}$
        \begin{itemize}
            \item $x = \texttt{nil}\in \Pi$ and $x\mapsto y\in \Sigma$. Inconsistent.
            \item $x = y \in \Pi $ and $\texttt{ls}(x,y)\in \Sigma$. Inconsistent.
            \item $x = y', z\ne y', x = z\in \Pi$. Inconsistent.
        \end{itemize}
        \item $\Pi\mid \Sigma \vdash E\ne F $ when $\texttt{Vars}'(E, F)=\emptyset$
        
        Add $E = F$ to the pure part and utilize analysis for \texttt{false}.
        
        \item $\Pi\mid \Sigma \vdash \texttt{Allocated}(E)$ when $ \texttt{Vars}'(E)=\emptyset$
    \end{itemize}
\end{example}
    
\end{frame}

\begin{frame}{Concrete and Symbolic Execution Semantic: Program Considered}
    \begin{center}
        \includegraphics[scale=0.45]{program.png}
    \end{center}
    We use $A(E)$ for any element from the set of primitive commands $\{x:=[E], [E]:=F, \mathbf{dispose}(E)\}$.
    
    With the existing result of \footnote{Josh Berdine et al. Symbolic Execution with Separation Logic}, the author provide the following execution semantics.
\end{frame}

\begin{frame}{Concrete Execution Semantic}
    \begin{center}
        \includegraphics[scale=0.4]{conc_rules.png}
    \end{center}
    The $\sigma$ and $\sigma'$ here in the concrete execution rules are the elements of \texttt{States} defined before.
    
    We extend it to $\mathcal{C}\llbracket c\rrbracket: \mathcal{P}(\mathtt{States})\rightarrow  \mathcal{P}(\mathtt{States})$
    \begin{example}
    An example for the mutation rule:
    
    $s(x) = 10, s(y) = 1$ and $h$ has definition on $10$.
    
    $s, h, [x] := y \Longrightarrow s, (h\mid 10\mapsto 1)$
    
    \end{example}
\end{frame}

\begin{frame}{Symbolic Execution Semantic}
    The symbolic execution semantic $\sigma,p\Longrightarrow \sigma'$, $\sigma$ here is a symbolic heap.
    \begin{center}
        \includegraphics[scale=0.5]{symb_exec.png}
    \end{center}
    These rules also results in an symbolic execution semantic function $\mathcal{I}\llbracket c\rrbracket: \mathcal{P(SH)\rightarrow P(SH)} $.
\end{frame}

\begin{frame}[fragile]{Symbolic Execution Semantic: Example}
\begin{center}
    \includegraphics[scale=0.35]{symb_exec.png}
\end{center}
\begin{example}
\begin{center}
    
\begin{multicols}{2}
\begin{lstlisting}
int main(){
    // true|emp
    x = 1;
    // x=x'|emp
    new(y);
    // x=x'|emp * y|->y'
    [y]=x;
    // x=x'|emp * y|->x
    dispose(y);
    // x=x'|emp
    dispose(y);
    // T
}
\end{lstlisting}
\end{multicols}
\end{center}
\end{example}


\end{frame}

\begin{frame}{Rearrangement Rules for \texttt{ls} Predicate}
    \begin{center}
        \includegraphics[scale=0.5]{rar.png}
    \end{center}
    
\end{frame}

\begin{frame}{Meaning Function and Safe Semantic}
    To state that the symbolic semantic is sound, we define a \emph{meaning function}:
    \[\gamma: \mathcal{P(SH)}\rightarrow \mathcal{P}(\mathtt{States})\]
    to maps the set of symbolic heap to the set of states that whose elements at least make one of the symbolic heap satisfied.
    
    Then by checking the rules we have
    \begin{theorem}
    The symbolic semantics is a sound overapproximation of the concrete semantics:
    \[\forall X\in \mathcal{P(SH)}. \mathcal{C}\llbracket c\rrbracket(\gamma(X)) \subseteq \gamma(\mathcal{I}\llbracket c\rrbracket X)\]
    
    \end{theorem}
\end{frame}

\begin{frame}{Problem Encountered}
    If there is no loop in the program, the analysis can be done by only using the symbolic execution on separation logic.
    \begin{center}
        \includegraphics[scale=0.4]{semantic.png}
    \end{center}
    The domain  $\mathcal{SH}$ is infinite, plus there is no limit on the primed variables:
    
    Define an abstract domain for the fix-point convergence and abstraction rules for the conversion.
    \begin{itemize}
        \item Expression replacement for primed variables.
        \item Garbage collection rules.
        \item List abstraction rules.
    \end{itemize}
    The conversion is given by $\leadsto$. The abstract domain after the abstraction is $\mathcal{CSH}$ and corresponding abstract semantic function $\mathcal{A}\llbracket c\rrbracket: \mathcal{P(CSH) \rightarrow P(CSH)}$.
\end{frame}

\begin{frame}{The Analysis}
    \begin{theorem}
    $\mathcal{CSH}$ is finite.
    \end{theorem}
    \begin{theorem}
    The abstract semantic is a sound overapproximation of the concrete semantic.\[\forall X\in \mathcal{P(SH)}. \mathcal{C}\llbracket c\rrbracket(\gamma(X)) \subseteq \gamma(\mathcal{A}\llbracket c\rrbracket X)\]
    \end{theorem}
    \begin{example}
    A program to dispose a list.
    
    \texttt{Program}: \textbf{while} $(c\ne0)$ \textbf{do} ($t:=c;c:=c\rightarrow tl;\mathbf{dispose(t)}$)
    
    \texttt{Pre:} $\{\}\mid\{\texttt{ls}(c,0)\}$
    
    \texttt{Post:} $\{c=0\}\mid \{\}$
    
    \texttt{Inv:} $\{c=0\}\mid \{\} \vee \{\}\mid \{\texttt{ls}(c,0)\}$
    \end{example}
\end{frame}

\begin{frame}{Locality}
    \begin{example}
    Program: $x:=c;c:=c\rightarrow tl$
    
    \texttt{Pre:} $\{\}\mid \{\texttt{ls}(c,d)*d\mapsto d'\}$

    \texttt{Post:} $\{c=d\}\mid \{x\mapsto d *d\mapsto d'\} \vee \{\}\mid \{x\mapsto c* \texttt{ls}(c,d) *d\mapsto d'\}$  
    
    A smaller input provide information we need:
    
    \texttt{Pre:} $\{\}\mid \{\texttt{ls}(c,d)\}$

    \texttt{Post:} $\{c=d\}\mid \{x\mapsto d \} \vee \{\}\mid \{x\mapsto c* \texttt{ls}(c,d) \}$  
    \end{example}
    
    The notion of $*$ on symbolic heaps:
    \[(\Pi_1\mid \Sigma_1) *(\Pi_2\mid \Sigma_2) = ((\Pi_1\cup \Pi_2\mid \Sigma_1*\Sigma_2))\]
    
    \begin{theorem}[Frame Rule]
    For all $X,Y\in \mathcal{P(CSH)}$, if $\texttt{Vars}(Y) \cap Mod = \emptyset$ then $\gamma(\mathcal{A}\llbracket c\rrbracket(X*Y)) \subseteq \gamma(\mathcal{A}\llbracket c\rrbracket X)*Y$
    \end{theorem}
\end{frame}
\end{document}