
\documentclass[aspectratio=1610, 13pt]{beamer}

\usepackage{xcolor}
\usepackage{multicol}
\usepackage{mathtools,array}
\usepackage[T1]{fontenc}
\usepackage{zi4}
\usepackage[font={scriptsize,bf}]{caption}
% \usepackage{subcaption}
\usepackage{graphics}
\usepackage{tikz}
\usepackage{fontawesome5}
\usepackage{mathpartir}

\newcommand{\naturals}{\mathbb{N}}
\newcommand{\reals}{\mathbb{R}}

\newcommand{\Dist}[1]{\mathcal{D}(#1)}
\newcommand{\expectation}{\mathbb{E}}

\newcommand{\states}{S}
\newcommand{\actions}{A}
\newcommand{\observables}{O}
\newcommand{\trans}{T}
\newcommand{\obs}{Z}
\newcommand{\reward}{R}
\newcommand{\discount}{\gamma}

\newcommand{\beliefs}{\mathcal{B}}
\newcommand{\beliefUpdate}{\tau}

\newcommand{\policy}{\pi}

\newcommand{\diff}[1]{\mathop{}\!\mathrm{d}#1}
\renewcommand{\figurename}{Figure}
\renewcommand{\refname}{Reference}

\AtBeginDocument{
  \catcode`_=12
  \begingroup\lccode`~=`_
  \lowercase{\endgroup\let~}\sb
  \mathcode`_="8000
}

% \usetheme{Madrid}
% % \usetheme{default}
% \setbeamertemplate{caption}[numbered]
% \setbeamerfont{title}{size=\large}
\mode<presentation>
{
  \usetheme{Darmstadt}      % or try Darmstadt, Madrid, Warsaw, ...
  \usecolortheme{default} % or try albatross, beaver, crane, ...
  \usefonttheme[onlymath]{serif}  % or try serif, structurebold, ...
  \setbeamertemplate{navigation symbols}{}
  \setbeamertemplate{caption}[numbered]
  \setbeamertemplate{footline}[frame number] 
} 

\usepackage{listings}
\lstdefinestyle{heaplang}{
    language=Caml,
    basicstyle=\footnotesize\ttfamily,
    keywordstyle=\color{blue},
    commentstyle=\color{red},
    escapeinside={<@}{@>},
    morekeywords={new_chan, fork, recv, send, swap, ref}
}
\lstdefinestyle{clang}{
    language=Caml,
    basicstyle=\footnotesize\ttfamily,
    keywordstyle=\color{blue},
    commentstyle=\color{red},
    escapeinside={<@}{@>},
}
\lstset{style=heaplang}

\usepackage{natbib}

\newcommand{\buchi}{B\"uchi }

\definecolor{goldenpoppy}{rgb}{0.99, 0.76, 0.0}
\definecolor{goldenyellow}{rgb}{1.0, 0.87, 0.0}
\definecolor{green2}{rgb}{0.1,0.7,0.3} 
\newcommand{\gcheck}{{\color{green2}\faCheckCircle[regular] }}
\newcommand{\rcross}{{\color{red} \faTimesCircle[regular]} }
\newcommand{\rflag}{{\color{red} \faFlag}}
% \usepackage{algorithm,amsmath}
% \usepackage[noend]{algpseudocode}

\newcommand{\zlstinline}{\let\par\endgraf\lstinline}
\newcommand{\comments}[1]{{\color{red}#1}}
\title{Group Meeting - 5}
\date{\today}
\author{Member: Yong Li, Depeng Liu, Weizhi Feng, Xie Li, Shizhen Yu, Yutian Zhu and Zongxin Liu}
\begin{document}
\maketitle

\begin{frame}\frametitle{MemSafety: Progress and Problems}
Previous plan: add implementation at SMACK frontend to support symbolic execution of separation logic.

Problems encountered: 
\begin{itemize}
\item How to express symbolic heap in data structure/generated boogie program? (Feasible)
\begin{itemize}
\item Added definition of symbolic heap into \texttt{BoogieAST.h}.

\end{itemize}

\item How to modify the symbolic heap? (Feasible)
\begin{itemize}
\item $\langle \Pi \mid \Sigma\rangle$, assume $\Sigma$ has a fix order during the execution, find pattern.


\end{itemize}
\item VC generation and verification (Problematic):
\begin{itemize}
\item Seems tedious to use Boogie IVL because it is hard to unified separation logic at backend. (TODO: a more thorough look into Boogie to find out capabilities.)
\item Only use SMACK frontend and implement our own VC generation. (Feasible)
\end{itemize} 
\item Interblock: construct the CFG during instruction visiting. (Feasible)

\end{itemize}

Plan: implement SE for \texttt{malloc,free} and assignment in one block.
\end{frame}


\begin{frame}{Shizhen Yu}
    \begin{itemize}
        \item  Reading
        \begin{itemize}
            \item  Enhancing Symbolic Execution of Heap-based Programs with Separation Logic for Test Input Generation
            \begin{itemize}
            \item Motivation and Illustration
            \item Operational semantics of the core language
            \end{itemize}
        \end{itemize}
        \item Preparing for matrix test
    %     \item \textbf{Bounded: }
    %     \begin{itemize}
        
    %     \item Definition of bounded \buchi automata and bounded languages.
   	% 	 \item Relationship of bounded languages and $\omega$-regular languages. 
    %     \end{itemize}
        
        \item Plan
        \begin{itemize}
            \item Continue reading "Details on symbolic execution using SL" part of "Enhancing Symbolic Execution of Heap-based Programs with Separation Logic for Test Input Generation"  
        \end{itemize}
    \end{itemize}
    
\end{frame}

\begin{frame}
  \frametitle{Yutian Zhu}
  \begin{itemize}
      \item Compositional Shape Analysis by means of Bi-Abduction(Done, prepare for ppt)
  \end{itemize}
  Recent:
  \begin{itemize}
    \item Prepare for tests. A lot of homework.
  \end{itemize}
  Plan:
  \begin{itemize}
    \item Static automated program repair for heap properties. ICSE 2018. Rijnard van Tonder, Claire Le Goues
\end{itemize}  

\end{frame}

\end{document}