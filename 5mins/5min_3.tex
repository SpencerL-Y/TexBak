\documentclass[aspectratio=1610]{beamer}

\usepackage[UTF8]{ctex}
\usepackage{xcolor}
\usepackage{multicol}
\usepackage{mathtools,array}
\usepackage[T1]{fontenc}
\usepackage{zi4}
\usepackage[font={scriptsize,bf}]{caption}
% \usepackage{subcaption}
\usepackage{graphics}
\usepackage{tikz}
\usepackage{fontawesome5}
\usepackage{mathpartir}

\newcommand{\naturals}{\mathbb{N}}
\newcommand{\reals}{\mathbb{R}}

\newcommand{\Dist}[1]{\mathcal{D}(#1)}
\newcommand{\expectation}{\mathbb{E}}

\newcommand{\states}{S}
\newcommand{\actions}{A}
\newcommand{\observables}{O}
\newcommand{\trans}{T}
\newcommand{\obs}{Z}
\newcommand{\reward}{R}
\newcommand{\discount}{\gamma}

\newcommand{\beliefs}{\mathcal{B}}
\newcommand{\beliefUpdate}{\tau}

\newcommand{\policy}{\pi}

\newcommand{\diff}[1]{\mathop{}\!\mathrm{d}#1}
\renewcommand{\figurename}{Figure}
\renewcommand{\refname}{Reference}

\AtBeginDocument{
  \catcode`_=12
  \begingroup\lccode`~=`_
  \lowercase{\endgroup\let~}\sb
  \mathcode`_="8000
}

% \usetheme{Madrid}
% % \usetheme{default}
% \setbeamertemplate{caption}[numbered]
% \setbeamerfont{title}{size=\large}
\mode<presentation>
{
  \usetheme{Darmstadt}      % or try Darmstadt, Madrid, Warsaw, ...
  \usecolortheme{default} % or try albatross, beaver, crane, ...
  \usefonttheme[onlymath]{serif}  % or try serif, structurebold, ...
  \setbeamertemplate{navigation symbols}{}
  \setbeamertemplate{caption}[numbered]
  \setbeamertemplate{footline}[frame number] 
} 

\usepackage{listings}
\lstdefinestyle{heaplang}{
    language=Caml,
    basicstyle=\footnotesize\ttfamily,
    keywordstyle=\color{blue},
    commentstyle=\color{gray},
    escapeinside={<@}{@>},
    morekeywords={new_chan, fork, recv, send, swap, ref}
}
\lstdefinestyle{clang}{
    language=Caml,
    basicstyle=\footnotesize\ttfamily,
    keywordstyle=\color{blue},
    commentstyle=\color{gray},
    escapeinside={<@}{@>},
}
\lstset{style=heaplang}

\usepackage{natbib}

\newcommand{\buchi}{B\"uchi }

\definecolor{goldenpoppy}{rgb}{0.99, 0.76, 0.0}
\definecolor{goldenyellow}{rgb}{1.0, 0.87, 0.0}
\definecolor{green2}{rgb}{0.1,0.7,0.3} 
\newcommand{\gcheck}{{\color{green2}\faCheckCircle[regular] }}
\newcommand{\rcross}{{\color{red} \faTimesCircle[regular]} }
\newcommand{\rflag}{{\color{red} \faFlag}}
% \usepackage{algorithm,amsmath}
% \usepackage[noend]{algpseudocode}

\newcommand{\zlstinline}{\let\par\endgraf\lstinline}
\newcommand{\comments}[1]{{\color{red}#1}}
\title{Group Meeting - 2}
\date{\today}
\author{Member: Yong Li, Depeng Liu, Weizhi Feng, Xie Li, Shizhen yu, Zongxin Liu}
\begin{document}
\maketitle

\begin{frame}\frametitle{MemSafety: Progress and Problem Encountered}

\begin{itemize}
\item SMACK: code reading.
\begin{itemize}
\item Problem 1: How SMACK do the region splitting to modify the memory model? 

Paper reading: 
\begin{itemize}
\item 
[{[1]}] Making Context-sensitive Points-to Analysis with Heap Cloning Practical For The Real World.
\item DSA, Andersen style analysis, Steensgaard's analysis and shape analysis.
\end{itemize}


\item Problem 2: How to adapt the assertion of separation logic into the generating and parsing of Boogie IVL.

Paper reading: 

\begin{itemize}
\item [{[2]}] A Primer on Separation Logic.
\item [{[3]}] Enhancing Symbolic Execution of Heap-based Programs with Separation Logic for Test Input Generation.
\end{itemize}

Tool investigating:

\begin{itemize}
\item Predator, on how they deal with separation assertions.
\end{itemize}

Case study: 
\begin{itemize}
\item Find a example program and write assertions in FOL and SL by hand to compare.
\end{itemize}


\item Problem 3: Sorting a document about the whole procedure of SMACK.
\end{itemize}
\end{itemize}
\end{frame}


\begin{frame}\frametitle{MemSafety: Progress and Problem Encourtered}
\begin{itemize}
\item SMACK: running SV-COMP cases.
\begin{itemize}
\item Problem 1: Specify the detailed result: which property, consistent or not.

\item Problem 2: Report the issue in Github.
\end{itemize}
\end{itemize}
\end{frame}

\section{Outline}
\begin{frame}{BBA: Outline}
    \begin{itemize}
        \item  Bounded \buchi automata.
        \item \textbf{Bounded: }
        \begin{itemize}
        
        \item Definition of bounded \buchi automata and bounded languages.
   		 \item Relationship of bounded languages and $\omega$-regular languages. 
        \end{itemize}
        \item Plan.
    \end{itemize}
    
\end{frame}

\section{Automata}
\begin{frame}{BBA: Bounded \buchi Automaton \& Bounded Language}
\begin{block}{Definition}
    Given an integer $d > 0$ and a \buchi automaton $\mathcal{A}$, we call the \buchi automaton with the integer $d$ as a bounded \buchi automaton.
\end{block}
\begin{block}{Definition}
    A run $\rho=q_{0}q_{1}...$ is accepting iff there exists an integer $i \geq 0$, the distance between any two consecutive accepting states with index greater than i is at most $d$. 
    
    Formally, a run is accepting iff $\exists i \geq 0, \forall j \geq i, \{q_{j},q_{j+1},...,q_{j+d-1}\} \cap F \ne \emptyset$, where F is the set of accepting states. Then we call such an accepting run a bounded run. A bounded word $w$ is accepted by $(\mathcal{A}, d)$ if there is an accepting bounded run of $(\mathcal{A}, d)$ on $w$. The bounded language recognized by $(\mathcal{A}, d)$, denoted $\mathcal{L(A,\textit{d})}$, is the set of bounded words that $(\mathcal{A},d)$ accepts. 
\end{block}

\end{frame}



\begin{frame}{BBA: Bounded Languages \& $\omega$-regular Languages}
    \begin{itemize}
        \item Proved that bounded languages are $\omega$-regular languages.
        \item Proved that $\omega$-regular languages cannot be expressed by bounded languages. 
        \item Therefore, bounded languages are the subset of $\omega$-regular languages.
    \end{itemize}
\end{frame}

\begin{frame}{BBA: Plan}
    Thinking about the computation of bounded languages...
    \begin{itemize}
        \item Intersection;
        \begin{itemize}
            \item Two bounded automata with the same $d$ and different $d$;
        \end{itemize}
        \item Complement.
    \end{itemize}
\end{frame}

\end{document}