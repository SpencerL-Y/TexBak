
\documentclass[aspectratio=1610, 13pt]{beamer}
\usepackage{ctex}
\usepackage{CJKutf8}
\usepackage{xcolor}
\usepackage{multicol}
\usepackage{mathtools,array}
\usepackage[T1]{fontenc}
\usepackage{zi4}
\usepackage[font={scriptsize,bf}]{caption}
% \usepackage{subcaption}
\usepackage{graphics}
\usepackage{tikz}
\usepackage{fontawesome5}
\usepackage{mathpartir}

\newcommand{\naturals}{\mathbb{N}}
\newcommand{\reals}{\mathbb{R}}

\newcommand{\Dist}[1]{\mathcal{D}(#1)}
\newcommand{\expectation}{\mathbb{E}}

\newcommand{\states}{S}
\newcommand{\actions}{A}
\newcommand{\observables}{O}
\newcommand{\trans}{T}
\newcommand{\obs}{Z}
\newcommand{\reward}{R}
\newcommand{\discount}{\gamma}

\newcommand{\beliefs}{\mathcal{B}}
\newcommand{\beliefUpdate}{\tau}

\newcommand{\policy}{\pi}

\newcommand{\diff}[1]{\mathop{}\!\mathrm{d}#1}
\renewcommand{\figurename}{Figure}
\renewcommand{\refname}{Reference}

\AtBeginDocument{
  \catcode`_=12
  \begingroup\lccode`~=`_
  \lowercase{\endgroup\let~}\sb
  \mathcode`_="8000
}

% \usetheme{Madrid}
% % \usetheme{default}
% \setbeamertemplate{caption}[numbered]
% \setbeamerfont{title}{size=\large}
\mode<presentation>
{
  \usetheme{Darmstadt}      % or try Darmstadt, Madrid, Warsaw, ...
  \usecolortheme{default} % or try albatross, beaver, crane, ...
  \usefonttheme[onlymath]{serif}  % or try serif, structurebold, ...
  \setbeamertemplate{navigation symbols}{}
  \setbeamertemplate{caption}[numbered]
  \setbeamertemplate{footline}[frame number] 
} 

\usepackage{listings}
\lstdefinestyle{heaplang}{
    language=C,
    basicstyle=\footnotesize\ttfamily,
    keywordstyle=\color{blue},
    commentstyle=\color{red},
    escapeinside={<@}{@>},
    morekeywords={new_chan, fork, recv, send, swap, ref}
}
\lstdefinestyle{clang}{
    language=C,
    basicstyle=\footnotesize\ttfamily,
    keywordstyle=\color{blue},
    commentstyle=\color{red},
    escapeinside={<@}{@>},
}
\lstset{style=heaplang}

\usepackage{natbib}

\newcommand{\buchi}{B\"uchi }

\definecolor{goldenpoppy}{rgb}{0.99, 0.76, 0.0}
\definecolor{goldenyellow}{rgb}{1.0, 0.87, 0.0}
\definecolor{green2}{rgb}{0.1,0.7,0.3} 
\newcommand{\gcheck}{{\color{green2}\faCheckCircle[regular] }}
\newcommand{\rcross}{{\color{red} \faTimesCircle[regular]} }
\newcommand{\rflag}{{\color{red} \faFlag}}
% \usepackage{algorithm,amsmath}
% \usepackage[noend]{algpseudocode}

\newcommand{\zlstinline}{\let\par\endgraf\lstinline}
\newcommand{\comments}[1]{{\color{red}#1}}
\title{Group Meeting - 7}
\date{\today}
\author{Members: Yong Li, Depeng Liu, Weizhi Feng, Xie Li, Shizhen Yu, Yutian Zhu, Zongxin Liu}
\begin{document}
\maketitle
\begin{frame}\frametitle{差分隐私}
这俩周:
\begin{itemize}
  \item 中期答辩;
  \item Aplas期刊:其他部分初步完成;
  由于MDP算法仅是充分的,Stream例子的pan-privacy分析可能有点问题,需加以解释和根据实验情况确定。
  \item 在调研几个部分:预估通过学习算法学得DP模型可行性、实际代码中的DP部署及验证、APPLE的差分隐私验证、隐私参数的选取等。
  \item 用SageMath做带有绝对值的连续噪声的带参定积分,分段积分是可计算的,解决简单连续机制如Laplace的验证;
  复杂机制刻画输入输出关系方面困难一些,需尝试。
\end{itemize}

计划:
\begin{itemize}
  \item Aplas期刊:计划完成
  \item 调研部分:报告讨论一下调研情况,再具体分析后续工作。
\end{itemize}
\end{frame}

\begin{frame}\frametitle{内存安全工具开发}

进展情况:
\begin{itemize}
\item 添加了对于\texttt{load}语句的支持。
\item 完成了对于一个基本块的符号执行、验证条件生成的框架,目前可以对一个基本块内的内存泄漏性质进行验证。
\end{itemize}
TODOs:
\begin{itemize}
\small
\item 修复指针算术bit和byte的bug (DONE)
\item 修复blk语义导致的entailment不成功问题 (DONE)
\item 修复blk分裂时,如果一开始malloc的大小是0的问题
\item 加入对数组和结构体的支持
\begin{itemize}
\item 对alloca指令进行相关的符号执行处理
\item 将所有变量在调用求解器之前转为字节为基本单位:需要在符号执行时所有变量的类型,等式翻译时的cast问题的处理,新旧变量之间的关系和新变量的取值约束
\end{itemize}

\item 完成对从SV-COMP改造过来的例子内存泄漏的分析。

\end{itemize}

\end{frame}

\begin{frame}[fragile]\frametitle{例子}
\begin{multicols}{2}
\begin{lstlisting}
typedef struct {
    void *lo;
    void *hi;
} TData;
\end{lstlisting}
\begin{lstlisting}
int main(){
    TData data;
    TData* pdata = &data;
    pdata->lo = malloc(16);
    pdata->hi = malloc(24);
    void *lo = pdata->lo;
    void *hi = pdata->hi;
    if (lo == hi) {
        free(lo);
        free(hi);
    }
    pdata->lo = (void *) 0;
    pdata->hi = (void *) 0;
}
\end{lstlisting}

\begin{itemize}
\item 可能有机会完成的内容:将\texttt{BlockExecutor}用到CFG上,实现CFG上的符号执行。
\end{itemize}
\end{multicols}

\end{frame}

\begin{frame}\frametitle{内存安全工具开发}
文献阅读:
\begin{itemize}

\item [{[1]}] Thomas Ströder et al. Proving Termination and Memory Safety for Programs with Pointer Arithmetic. IJCAR'14.
\end{itemize}
\end{frame}
\begin{frame}{冯维直}
进展:
    \begin{itemize}
        \item 阅读rank-based和slice-based算法做unambiguous \buchi automata取补的GandALF‘20文章,在李老师的指导下基于spot库实现slice-based的算法.
        \item [-] 结合NCSB算法的TACAS文章读seminator和spot中实现semi-deterministic \buchi automata取补的代码.
        \item [-] 基于seminator和spot提供的自动机库和接口等实现了unambiguous \buchi automata取补的slice-based的算法,运行结果有些问题,需要继续调试.
\end{itemize}    
计划:
    
    \begin{itemize}
        \item Slice-based算法正确实现成功运行,进行实验,和李老师讨论,在GandALF‘20文章基础上写实验部分的章节.
\end{itemize}    
\end{frame}


\end{document}