
\documentclass[aspectratio=1610, 13pt]{beamer}
\usepackage{ctex}
\usepackage{CJKutf8}
\usepackage{xcolor}
\usepackage{multicol}
\usepackage{mathtools,array}
\usepackage[T1]{fontenc}
\usepackage{zi4}
\usepackage[font={scriptsize,bf}]{caption}
% \usepackage{subcaption}
\usepackage{graphics}
\usepackage{tikz}
\usepackage{fontawesome5}
\usepackage{mathpartir}

\newcommand{\naturals}{\mathbb{N}}
\newcommand{\reals}{\mathbb{R}}

\newcommand{\Dist}[1]{\mathcal{D}(#1)}
\newcommand{\expectation}{\mathbb{E}}

\newcommand{\states}{S}
\newcommand{\actions}{A}
\newcommand{\observables}{O}
\newcommand{\trans}{T}
\newcommand{\obs}{Z}
\newcommand{\reward}{R}
\newcommand{\discount}{\gamma}

\newcommand{\beliefs}{\mathcal{B}}
\newcommand{\beliefUpdate}{\tau}

\newcommand{\policy}{\pi}

\newcommand{\diff}[1]{\mathop{}\!\mathrm{d}#1}
\renewcommand{\figurename}{Figure}
\renewcommand{\refname}{Reference}

\AtBeginDocument{
  \catcode`_=12
  \begingroup\lccode`~=`_
  \lowercase{\endgroup\let~}\sb
  \mathcode`_="8000
}

% \usetheme{Madrid}
% % \usetheme{default}
% \setbeamertemplate{caption}[numbered]
% \setbeamerfont{title}{size=\large}
\mode<presentation>
{
  \usetheme{Darmstadt}      % or try Darmstadt, Madrid, Warsaw, ...
  \usecolortheme{default} % or try albatross, beaver, crane, ...
  \usefonttheme[onlymath]{serif}  % or try serif, structurebold, ...
  \setbeamertemplate{navigation symbols}{}
  \setbeamertemplate{caption}[numbered]
  \setbeamertemplate{footline}[frame number] 
} 

\usepackage{listings}
\lstdefinestyle{heaplang}{
    language=C,
    basicstyle=\footnotesize\ttfamily,
    keywordstyle=\color{blue},
    commentstyle=\color{red},
    escapeinside={<@}{@>},
    morekeywords={new_chan, fork, recv, send, swap, ref}
}
\lstdefinestyle{clang}{
    language=C,
    basicstyle=\footnotesize\ttfamily,
    keywordstyle=\color{blue},
    commentstyle=\color{red},
    escapeinside={<@}{@>},
}
\lstset{style=heaplang}

\usepackage{natbib}

\newcommand{\buchi}{B\"uchi }

\definecolor{goldenpoppy}{rgb}{0.99, 0.76, 0.0}
\definecolor{goldenyellow}{rgb}{1.0, 0.87, 0.0}
\definecolor{green2}{rgb}{0.1,0.7,0.3} 
\newcommand{\gcheck}{{\color{green2}\faCheckCircle[regular] }}
\newcommand{\rcross}{{\color{red} \faTimesCircle[regular]} }
\newcommand{\rflag}{{\color{red} \faFlag}}
% \usepackage{algorithm,amsmath}
% \usepackage[noend]{algpseudocode}

\newcommand{\zlstinline}{\let\par\endgraf\lstinline}
\newcommand{\comments}[1]{{\color{red}#1}}
\title{Group Meeting - 9}
\date{\today}
\author{Members: Yong Li, Depeng Liu, Weizhi Feng, Xie Li, Shizhen Yu, Yutian Zhu, Zongxin Liu}
\begin{document}
\maketitle

\begin{frame}\frametitle{差分隐私}
这周:
\begin{itemize}
  \item Aplas期刊:已用PRISM求解差分隐私问题;针对部分问题,为了防止出现计算Pmin=0的情况,加入公平性性质求得更精确的隐私参数;
  文章再和老师一起加工一下准备这两天投稿;
  \item 读formal testing的文章,准备和蓝牙协议的黑盒模型学习一起报告。
\end{itemize}

计划:
\begin{itemize}
  \item Aplas期刊:计划完成;
  \item 报告讨论一下文章,讨论后续差分隐私模型的学习问题;
  \item 准备几个报告(包括学习、这周五和下周三组会)。
\end{itemize}
\end{frame}

\begin{frame}\frametitle{内存安全工具开发}

上周目标和进展情况:
\begin{itemize}
\item 支持在单个CFG上的符号执行(已完成,循环展开)
\item \texttt{memcpy()}和\texttt{memset()}的语义(存在一些问题)

\item 其他细节问题:全局变量在Boogie中的处理…
\end{itemize}
TODOs:
\begin{itemize}

\item 过程间分析的调研和实现
\begin{itemize}
\item 实现直接内联的简单版本

\item 调研其他过程间分析的技术
\end{itemize}

\item 对标SV-COMP进行相关的函数语义的支持,运行更多例子。
\end{itemize}

 \end{frame}
 
 
\begin{frame}[fragile]\frametitle{内存安全工具的开发}

\begin{multicols}{2}
 \begin{lstlisting}
    int a = 10;
    int* j = malloc(4);
    if(a > 10){
        free(j);
    }
 \end{lstlisting}
 \begin{lstlisting}
    for(int i = 0; i < 15; i ++){
        int *j = (int*)malloc
        (sizeof(int));
        if(i < 10){
            free(j);
        }
    }
 
 \end{lstlisting}
 
 \begin{lstlisting}
    TData data;
    TData* pdata = &data;

    TData c;
    pdata->lo = malloc(16);
    pdata->hi = malloc(24);
    void *lo = pdata->lo;
    void *hi = pdata->hi;
    if(lo == hi){
        free(lo);
        free(hi);
    }
 
 \end{lstlisting}
 
 
 \end{multicols}
 
 
 
\end{frame}
 
\begin{frame}[fragile]\frametitle{内存安全工具的开发}
\begin{lstlisting}
    n = 128;
    a = malloc (n * sizeof(*a));
    b = malloc (n * sizeof(*b));
    *b++ = 0;
    int i;
    for (i = 0; i < n; i++)
        a[i] = -1;
    for (i = 0; i < 128 - 1; i++)
        b[i] = -1;
    if (b[-2]) /* invalid deref */
    { free(a); free(b-1); }
    else
    { free(a); free(b-1); }

\end{lstlisting}
\end{frame}
 

\begin{frame}{冯维直}
进展:
    \begin{itemize}
        \item 代码实现:
        \item [-] 上周已经实现了GandALF文章中的NCB算法(一种对unambiguous \buchi automata取补的slice-based的算法).
        \item [-] 上周提到在实现李老师提出的semi-deterministic automata取补新算法时(NSBC算法)发现有些样例不对,本周已经将bug修复.
        \item [-] 读李老师博士论文中介绍pldi文章对semi-deterministic取补算法的改进,和李老师讨论了他的改进算法.
        \item [-] 目前在用seminator自带的公式和李老师给的几百个例子进行实验.
        
\end{itemize}    
计划:
    \begin{itemize}
        \item 目前的实验结果显示实现的NSBC算法优势不明显,需要进行优化或者找其他benchmark.
        \item 调研硬件验证的内容.
        \begin{itemize}
            \item [-] 问题,验证方法,难点,工具框架.
        \end{itemize}
\end{itemize}    
\end{frame}

\begin{frame}\frametitle{朱雨田:进度}
\begin{itemize}

\item Compositional Shape Analysis by means of Bi-Abduction - 整理完成
\item VeriSolid Correct-by-Design Smart Contracts for Ethereum FC2019 - 读完
\item Static Automated Program Repair for Heap Properties - 在读
\item 学习Ocaml
\end{itemize}
\end{frame}





\end{document}