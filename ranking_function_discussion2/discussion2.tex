\documentclass[11pt]{beamer}
\usepackage{verbatim}
\usepackage{amsmath}
\usepackage{amsthm}
\usepackage{graphics}
\usepackage{color}
\usepackage{stmaryrd}\usefonttheme[onlymath]{serif}

\title{Discussion}
\date{\today}

\begin{document}
\maketitle
\begin{frame}


\begin{frame}\frametitle{Crucial Problem}
Different from the disjunctive invariant problem, which only needs to generate a max-plus convex to surround the samples, the dependcy of different part is much more important.

Idea: using samples to infer the possible area to split the space.

\end{frame}

\end{document}