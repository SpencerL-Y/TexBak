\documentclass[11pt]{beamer}
\usepackage{verbatim}
\usepackage{amsmath}
\usepackage{amsthm}
\usepackage{graphics}
\usepackage{color}
\usepackage{stmaryrd}\usefonttheme[onlymath]{serif}

\title{On Multiphase-Linear Ranking Functions}
\date{\today}
\author{Amir M. Ben-Amram and Samir Genaim}
\newif\ifcomm\commfalse
\begin{document}
\maketitle


\begin{frame}\frametitle{Single Path Linear Constraint Loop}{\scriptsize
\begin{example}
\begin{center}

\includegraphics[scale = 0.4]{loopExample.png}


\end{center}


Let $B = (-1, 0, 1)$, $\textbf{x} = (x, y, z)^T, \textbf{b} = 0$.

Let $\textbf{x}'' = (x, y, z, x', y', z')$, 

\begin{equation}
A = \left[
\begin{array}{cccccc}
     1 &1 &0 -1 & 0 & 0 \\
     0 & 1 & 1 & 0 & -1 & 0 \\
     0& 0 & 1 & 0 & 0 & -1
\end{array}
\right]
\end{equation}

and $\textbf{c} = (0, 0, 1)^T$


\end{example}




%\end{frame}
%\begin{frame}\frametitle{Single Path Linear Constraint Loop}
\begin{definition}[SLC]
\includegraphics[scale = 0.28]{1.PNG}


\end{definition}
\begin{center}
\includegraphics[scale = 0.25]{2.PNG}

$A''\textbf{x}'' \le \textbf{c}''$
\end{center}
}
\end{frame}
\begin{frame}\frametitle{Ranking Functions}

\begin{definition}[Single Linear Ranking Function(LRF)]

$f(x_1, \ldots, x_n) = a_1x_1 + \ldots a_nx_n + a_0$, such that

\begin{itemize}
\item $f(\textbf{x}) \ge 0$ for any $\textbf{x}$ satisfies the loop constraints.

\item $f(\textbf{x}) - f(\textbf{x}') \ge 1$ for any transition from $\textbf{x}$ to $\textbf{x}'$.



\end{itemize}
\end{definition}

\begin{example}
\[\texttt{while }( x - 1 > 0) \texttt{do } x' = x - 5\]

LRF: $f(x) = ax + b$.
\begin{itemize}
\item $ax + b \ge 0$ $\Rightarrow$ $x \ge -\frac{b}{a} = 1$.
\item $ax + b - (ax' + b) = a(x - x') = 5a$ $\Rightarrow$ $5a \ge 1$
\end{itemize}
A possible SLRF: $f(x) = x - 1$
\end{example}

We can define a binary relation $\textbf{x} \succeq \textbf{x}'$ iff  $f(\textbf{x}) - f(\textbf{x}') \ge 1$ and $f(\textbf{x}) \ge 0$

\end{frame}



\begin{frame}\frametitle{Limitation of SLRF}
\[\texttt{while } (q > 0) \texttt{do } q' = q - y, y ' = y + 1\]

Assume there is a LRF for this loop, say $f(q, y) = a_1 q + a_2 y + b$

\[f(q, y) - f(q', y') = a_1y + a_2\]


Since $y$ is not bounded, we cannot guarantee  $\Delta f(q,y,q',y') > 0$

The loop does not has a SLRF, however, it does terminate.

We still wish to use $q$ for ranking function, but to distinguish different ``phase'' of $q$ base on either $y \ge 0$ or $y < 0$ 


\end{frame}
\begin{frame}\frametitle{Nested RF}

\begin{definition}[Nested Ranking Function]

A tuple $\langle f_1, \ldots, f_d\rangle$ is a nested ranking function for $T$ if the following requirements are satisfied for all $\textbf{x}''\in T$
\begin{center}
\includegraphics[scale = 0.3]{6.PNG}

\end{center}

Let $f_0 = 0$.
\end{definition}


\end{frame}


\begin{frame}\frametitle{Example: Nested RF}
\begin{center}
\includegraphics[scale = 0.3]{6.PNG}

\end{center}

\[\texttt{while } (q > 0) \texttt{do } q' = q - y, y ' = y + 1\]

\begin{itemize}
\item Above loop has Nested RF $\langle 1-y, q + 1\rangle$
\item 
\begin{center}
\includegraphics[scale = 0.35]{10.PNG}
\end{center}
\end{itemize}
\end{frame}
\begin{frame}\frametitle{Linear Loop Program}
\begin{definition}
A linear loop program \texttt{LOOP}$(x, x')$ is a binary relation defined by a formula with the free variables $x$ and $x'$ of the form
\begin{center}
\includegraphics[scale = 0.2]{15.PNG}

\end{center}
for some finite index set $I$.
\end{definition}
\begin{example}
\[\texttt{while } (q > 0)\{\texttt{if } (y > 0): q' = q - y - 1; \texttt{else }: q' = q + y - 1\}\]
can be represented by 
\begin{center}

\includegraphics[scale = 0.35]{17.PNG}
\end{center}
\end{example}
\end{frame}
\begin{frame}\frametitle{Limitation of Nested RF}
\begin{example}
\[\texttt{while }(q > 0 \vee y > 0) \]
\[\{\texttt{if }(y > 0): y' = y - 1; q ' = q;\texttt{else }: q' = q - 1\}\]
\end{example}
This program does not have a nested ranking function for we require $f_d\ge 0$ but the guard is $q > 0 \vee y > 0$.

Howerver, this loop does terminate. Then we use a ``multi-phase" ranking function $\langle y,  q\rangle$ to prove the termination.


\includegraphics[scale = 0.35]{6.PNG}
\end{frame}



\begin{frame}{Multiphase Ranking Function}
\begin{definition}
Given a set of transitions $T\subseteq \mathbb{Q}^{2n}$, we say $\langle f_1, \ldots, f_d\rangle$ is a multiphase ranking function for $T$ if for every $\textbf{x}'' \in T$, there is an index $i\in [1, d]$, s.t.

\begin{center}
\includegraphics[scale = 0.3]{3.PNG}
\end{center}
We say that $\textbf{x}''$ is ranked by $f_i$(for the minimal).
\end{definition}


\end{frame}


\begin{frame}\frametitle{Example: Multiphase Ranking Function}
\[\texttt{while }( x > -z) \texttt{do } x' = x + y, y' = y + z, z = z - 1\]

Attempt to use a ranking function that has several phases: 
$\langle z + 1, y + 1, x\rangle$
\begin{center}
\begin{tabular}{|c|c|c|c|c|c|}
\hline 
$x$&$y$&$z$&$z+1$&$y+1$&$x$\\
\hline
$1$&$1$&$1$&\textbf{2}&$2$&$1$\\
$2$&$2$&$0$&\textbf{1}&$3$&$2$\\
$4$&$2$&$-1$&\textbf{0}&$3$&$4$\\
\hline
$6$&$1$&$-2$&$-1$&\textbf{2}&$6$\\
$7$&$-1$&$-3$&$-2$&\textbf{0}&$7$\\
\hline
$6$&$-4$&$-4$&$-3$&$-3$&\textbf{6}\\
$2$&$-8$&$-5$&$-4$&$-7$&\textbf{2}\\
\hline
$-6$&$-13$&$-6$&$-5$&$-12$&$-6$\\
\hline
\end{tabular}
\end{center}
\end{frame}
\begin{frame}\frametitle{Example: Multiphase Ranking Function}
\[\texttt{while }( x > -z) \texttt{do } x' = x + y, y' = y + z, z' = z - 1\]
\[\langle z + 1, y + 1, x \rangle\]
$\textbf{x}''$ is ranked by $f_k$ when $i = k$. In this example, $f_1(x, y, z) = z + 1$, $f_2(x, y, z) = y + 1$ and $f_3(x, y, z) = x$
\begin{center}
\includegraphics[scale = 0.2]{3.PNG}

\includegraphics[scale = 0.28]{divide.png}
\end{center}
\end{frame}



\begin{frame}\frametitle{M$\Phi$RF to Nested RF}
\begin{theorem}[1]
If $\mathcal{Q}$ has a M$\Phi$RF of depth $d$, then it has a nested ranking function of depth at most $d$.


\end{theorem}




\end{frame}

\ifcomm
\begin{frame}\frametitle{M$\Phi$RF to Nested RF}
\begin{theorem}[1]
If $\mathcal{Q}$ has a M$\Phi$RF of depth $d$, then it has a nested ranking function of depth at most $d$.


\end{theorem}


\begin{proof}
By induction on the depth $d$.
\begin{itemize}
\item $d = 1$: M$\Phi$RF and nested RF are both LRF.
\item $d > 1$: $d = 2$ e.g.
$\langle f_1, f_2\rangle$. When index $i = 1$, we do not impose bound on $f_2(\textbf{x})$. However, a bound is needed for $f_2'(\textbf{x})$ in nested RF $\langle f_1', f_2'\rangle$.

To solve the problem that $f_2(\textbf{x})$ might goes under $0$, when $\textbf{x}''$ is ranked by $f_1$. Consider  $\mathcal{Q}' = \mathcal{Q}\cap \{\textbf{x}''\in \mathbb{Q}^{2n}\mid f_1(\textbf{x}) \le 0\}$


\end{itemize}


\end{proof}

\end{frame}

\begin{frame}\frametitle{M$\Phi$RF to Nested RF}
\begin{lemma}[1]
Let $\tau = \langle f_1, \ldots, f_d \rangle$ be an irredundant M$\Phi$RF for $\mathcal{Q}$, such that $\langle f_2, \ldots, f_d\rangle$ is a nested ranking function for $\mathcal{Q}' = \mathcal{Q}\cap \{\textbf{x}''\in \mathbb{Q}^{2n}\mid f_1(\textbf{x}) \le 0\}$. Then there is a nested ranking function of depth $d$ for $\mathcal{Q}$.

\end{lemma}
Prove by construction: construct a nested RF $\langle f_1', \ldots, f_d'\rangle$
\begin{center}
\includegraphics[scale = 0.3]{1.pdf}
\includegraphics[scale = 0.2]{6.PNG}


\end{center}
If $f_d$ is non-negative on $\mathcal{Q}$, then $f_d' = f_d$.
Otherwise, $\textbf{x}'' \in \mathcal{Q} \rightarrow f_d(\textbf{x}) \ge 0 \vee f_1(\textbf{x}) > 0$


\end{frame}

\begin{frame}
\begin{lemma}[0]
Given an non-empty polyhedron $\mathcal{P}$ and linear functions $f_1, \ldots , f_k$ such that 

\begin{enumerate}
\item $\textbf{x}\in \mathcal{P}\rightarrow f_1(  \textbf{x}) \ge 0 \vee \ldots \vee f_{k-1}(\textbf{x}) \ge 0 \vee f_{k}(\textbf{x}) \ge 0$

\item $\textbf{x} \in \mathcal{P} \not\rightarrow f_1(\textbf{x}) \ge 0 \vee \ldots\vee f_{k-1}(\textbf{x}) \ge 0$

\end{enumerate}
There exists a non-negative constants $\mu_1, \ldots, \mu_{k-1}$ such that 

\[\textbf{x}\in \mathcal{P} \rightarrow \mu_1f_1(\textbf{x}) + \ldots + \mu_{k-1}f_{k-1}(\textbf{x}) + f_k(\textbf{x}) \ge 0\]
\end{lemma}


\end{frame}

\begin{frame}\frametitle{Motzkin's Transposition Theorem}
\begin{theorem}[Motzkin's Transposition Theorem]
For $A\in \mathbb{K}^{m\times n}, C\in \mathbb{K}^{l\times n}, b\in \mathbb{K}^m, $ and $d\in \mathbb{K}^l$. The formulae below are equivalent.
\begin{itemize}
\item $\forall x \in \mathbb{K}^n. \neg (Ax\le b \wedge Cx < d)$

\item $\exists \lambda \in \mathbb{K}^m. \exists \mu \in \mathbb{K}^l.$

$\lambda \ge 0 \wedge \mu \ge 0$

$\wedge \lambda^{T}A + \mu^TC = 0 \wedge \lambda^{T}b + \mu^Td \le 0 $

$\wedge (\lambda^Tb < 0 \vee \mu \ne 0)$

\end{itemize}
\end{theorem}
\end{frame}


\begin{frame}\frametitle{Proof of Lemma (0)}
Let $\mathcal{P}$ be $B\textbf{x} \le c$, $f_i = \vec{a}_i \textbf{x} - b_i$.
Then $(i)$ is equivalent to the infeasibility of $B\textbf{x} \le c \wedge A\textbf{x} < b$, by the theorem we have $\vec{\lambda}, \vec{\mu}$ s.t.

\begin{center}
\includegraphics[scale=0.45]{11.png}
\end{center}
Then for all $\textbf{x}\in \mathcal{P}$
\begin{center}
\includegraphics[scale=0.35]{12.png}
\end{center}
\end{frame}

\begin{frame}\frametitle{M$\Phi$RF to Nested RF}
\begin{center}
\includegraphics[scale = 0.2]{6.PNG}
\end{center}

Assume $f_{n}'(\textbf{x})= f_{n}(\textbf{x}) + \mu_{n}f_1(\textbf{x})$ and $f_d', \ldots, f_{i}'$ has already been computed.
\[\textbf{x}''\in \mathcal{Q}' \rightarrow (\Delta f_i(\textbf{x}'') - 1) + f_{i-1}(\textbf{x}) \ge 0\]
\[\textbf{x}''\in \mathcal{Q}' \rightarrow (\Delta f_i'(\textbf{x}'') - 1) + f_{i-1}(\textbf{x}) \ge 0\]
If above inequation also holds for $\mathcal{Q}$, then $f_{i-1}' = f_{i-1}$, Otherwise
\[\textbf{x}''\in \mathcal{Q} \rightarrow (\Delta f_i'(\textbf{x}'') - 1) + f_{i-1}(\textbf{x}) \ge 0 \vee f_1(\textbf{x}) \ge 0\]


\end{frame}
\fi
\begin{frame}\frametitle{BM$\Phi$RF($\mathbb{Q})\in$\texttt{PTIME}}
\begin{theorem}[2]
BM$\Phi$RF($\mathbb{Q}$)$\in$\texttt{PTIME}.



\end{theorem}

\begin{proof}
Leike et al..Ranking Templates for Linear Loops. 

\end{proof}
\end{frame}

\iftrue
\begin{frame}\frametitle{LLRF}

Intuition: remind binary relation $\textbf{x} \succeq \textbf{x}'$ iff  $f(\textbf{x}) - f(\textbf{x}') \ge 1$ and $f(\textbf{x}) \ge 0$.

Generalize it into several phases using lexicographical order of ranking functions.

$\langle f_1, f_2, \ldots, f_d\rangle$

$(2,3,1,3) \ge (2,1,5,4)$

\begin{definition}[LLRF]
Given a set of transitions $T$ we say that 
$\langle f_1, f_2, \ldots, f_d\rangle$ is a LLRF (of depth $d$) for $T$ if for every $\textbf{x}''\in T$ there is an index $i$ such that 
\begin{center}
\includegraphics[scale = 0.26]{4.PNG}

\end{center}
A LLRF is weak if..
\end{definition}

\end{frame}
\begin{frame}\frametitle{Weak LLRF to M$\Phi$RF}

\begin{theorem}[3]
If $\mathcal{Q}$ has a weak LLRF of depth $d$, it has a  M$\Phi$RF of depth $d$.


\end{theorem}

\end{frame}
\ifcomm
\begin{frame}\frametitle{Weak LLRF to M$\Phi$RF}

\begin{theorem}[3]
If $\mathcal{Q}$ has a weak LLRF of depth $d$, it has a  M$\Phi$RF of depth $d$.


\end{theorem}

\begin{proof}
Prove by induction.

\begin{itemize}

\item $d = 1$: 
For LLRF: $\Delta f_1(\textbf{x}'') > 0$, $f_1(\textbf{x}) \ge 0$ is a LRF due to the loop is linear.

For M$\Phi$RF: is a LRF.

\item $d > 1$: Observe that for a given LLRF $\langle f_1, f_2, \ldots, f_d\rangle $, after removing $f_k$, $\langle f_1, \ldots, f_{k-1}, f_{k+1}, \ldots, f_d\rangle$ is also a LLRF.

If we apply IH here, we get a M$\Phi$RF of depth $d-1$.

\end{itemize}


\end{proof}


\end{frame}


\begin{frame}\frametitle{Weak LLRF to M$\Phi$RF}
Now we want some techniques to use a M$\Phi$RF of depth $d - 1$ and $f_k$ we removed to prove there is a M$\Phi$RF of depth $d$ on $\mathcal{Q}$. 
\begin{lemma}[2]
Let $f$ be a non-negative linear function over $\mathcal{Q}$. If $\mathcal{Q}' = \mathcal{Q}\cap \{\textbf{x} '' \mid \Delta f(\textbf{x} '') \le 0\}$ has a M$\Phi$RF of depth $d$, then $\mathcal{Q}$ has a M$\Phi$RF of depth at most $d + 1$.
\end{lemma}
\begin{proof}
Prove by construction: if the known M$\Phi$RF is $\langle g_1, \ldots, g_d \rangle$ and the funtion non-negative function is $f$, we wish to construct a  M$\Phi$RF $\langle g_1', \ldots, g_n', f\rangle$ of depth $d + 1$
\[\textbf{x}''\in \mathcal{Q} \rightarrow \Delta f(\textbf{x}'') > 0 \vee \Delta g_1(\textbf{x}'') \ge 1\]

Then update $g_2$ to $g_2'$ when $g'_1(\textbf{x}) \le -1$, an so on...

\end{proof}

\end{frame}

\begin{frame}\frametitle{Weak LLRF to M$\Phi$RF}
Remind the $f_k$ we removed in the theorem, together with Lemma(2), we wish to construct a non-negative linear function $g$ over $\mathcal{Q}$ and $g$ decrease on (at least) the same transitions of $f_k$.
\begin{lemma}[3]
Let $\langle f_1, \ldots, f_d\rangle$ be a weak LLRF for $\mathcal{Q}$. There is a linear function $g$ that is positive over $\mathcal{Q}$, and decreasing on (at least) the same transitions of $f_i$, for some $i\in [1,d]$.

\end{lemma}

\begin{proof}
Use Lemma(0) to find the $i$.
\end{proof}

\end{frame}
\begin{frame}\frametitle{M$\Phi$RF and LLRFs over the Integers}
%\begin{example}
%\[\texttt{while } (x_2 - x_1 \le 0, x_1 + x_2 \ge 1, x_3 \ge 0) 
%\]
%\[\texttt{do }x_2 ' = x_2 - 2x_1 + 1; x_3 ' = x_3 + 10x_2 + 9\]
%\end{example}
%Intepreted over integer: the loop has the M$\Phi$RF $\langle 10x_2, x_3\rangle$

%Intepreted over rationals: the loop does not terminate: $(\frac{1}{2}, \frac{1}{2}, 0)$

\begin{itemize}
\item Actual programs with \texttt{int}.

\item More important, conclusions for rational does not always applicable in on integer version.

\end{itemize}
\begin{example}
\begin{center}

\includegraphics[scale = .3]{7.PNG}

\end{center}
\end{example}
For rationals: $x_1 = \dfrac{1}{2}, x_2 = \dfrac{1}{2}$

For integers: there exists a linear ranking function $f(x_1, x_2) = x_1 + x_2$
\end{frame}

\begin{frame}\frametitle{M$\Phi$RF and LLRFs over the Integers}

\begin{itemize}
\item Integer case for LRF: completeness for the integer version was achieved by reducing the problem to the rational case. Intuitionly, since $\mathcal{Q}_I$ is the convex combinition of points in $I(\mathcal{Q})$.

\item Theorems above does not apply to the integer versions, but in the following we will prove that the reduction from integer case to rational also works for LLRF and M$\Phi$RF.
\end{itemize}


\end{frame}
\fi

\begin{frame}\frametitle{Weak LLRF: Integer to Rational}
\begin{theorem}[4]

Let $\langle f_1, \ldots, f_d\rangle$ be a weak LLRF for $I(\mathcal{Q})$. Then there are constants $c_1, \ldots, c_d$ such that $\langle f_1 + c_1, \ldots, f_d + c_d\rangle$ is a weak LLRF for $\mathcal{Q}_I$(over the rationals).

\end{theorem}

\end{frame}
\ifcomm
\begin{frame}\frametitle{Weak LLRF: Integer to Rational}
\begin{theorem}[4]

Let $\langle f_1, \ldots, f_d\rangle$ be a weak LLRF for $I(\mathcal{Q})$. Then there are constants $c_1, \ldots, c_d$ such that $\langle f_1 + c_1, \ldots, f_d + c_d\rangle$ is a weak LLRF for $\mathcal{Q}_I$(over the rationals).

\end{theorem}


\begin{proof}
prove by induction:

\begin{itemize}
\item $d = 1$, LRF.
\item $d > 1$, define 
\[\mathcal{Q}' = \mathcal{Q}_I\cap \{\textbf{x}''\mid f_1(\textbf{x}) \le -1\}\]
\[\mathcal{Q}'' = \mathcal{Q}_I \cap \{\textbf{x}''\mid \Delta f_1(\textbf{x}'') = 0\}\]

by IH, the theorem holds on $\mathcal{Q}_I'$ and $\mathcal{Q}_I''$ for weak LLRF of depth $d - 1$. say,
\[\langle f_2 + c_2', \ldots, f_d + c_d'\rangle, \langle f_2 + c_2'', \ldots, f_d + c_d''\rangle\]
\end{itemize}

\end{proof}
\end{frame}


\begin{frame}\frametitle{Proof Continue}


\[\mathcal{Q}' = \mathcal{Q}_I\cap \{\textbf{x}''\mid f_1(\textbf{x}) \le -1\}\]
\[\mathcal{Q}'' = \mathcal{Q}_I \cap \{\textbf{x}''\mid \Delta f_1(\textbf{x}'') = 0\}\]
Then we wish to have a lower bound on $f_1(\textbf{x})$.

\[\langle f_1 + c_1, f_2 + \max(c_2', c_2''), \ldots, f_d + \max(c_d', c_d'')\rangle\]

This implies the above formula is a weak LLRF on $\mathcal{Q}_I$. i.e. given a $\textbf{x}''\in \mathcal{Q}_I$, either..., or...

Problem: how to prove the existence of the lower bound?

\end{frame}


\begin{frame}\frametitle{Prove the Lower Bound}
$\mathcal{Q}_I', \mathcal{Q}_I''$.
\begin{itemize}
\item If $\mathcal{Q}_I'$ is empty, then by the definition of $\mathcal{Q}'$ $f_1$ is lower bounded.

\item Otherwise, prove the lower bound on $\mathcal{Q}_I \setminus \mathcal{Q}_I'$
\[\mathcal{Q}'_I = \{\textbf{x}'' \mid \vec{a}_i\cdot \textbf{x}'' \le b_i, i \in [1, m]\}\]
\[\mathcal{P}_i = \mathcal{Q}_I \cap \{\textbf{x}''\mid \vec{a}_i \cdot \textbf{x}'' > b_i\}\]
\[\mathcal{P}_i' = \mathcal{Q}_I \cap \{\textbf{x}''\mid \vec{a}_i \cdot \textbf{x}'' \ge b_i\}\]

For $i\in [1, m]$.
Then clearly $\mathcal{Q}_I\setminus \mathcal{Q}_I' \subseteq \bigcup_{i = 1}^{m} \mathcal{P}_i$, by construction all the integer points in $\mathcal{P}_i$ are also in $\mathcal{Q}_I\setminus \mathcal{Q}_I'$.

Proof target: for every $i$, $f_1$ is lower bounded in $\mathcal{P}_i$ for every $i$.

Fix $i$ for the following arguments, s.t. $\mathcal{P}_i$ is not empty.



\end{itemize}


\end{frame}

\begin{frame}\frametitle{Intuition: Proof of the Lower Bound}
\[\mathcal{Q}' = \mathcal{Q}_I\cap \{\textbf{x}''\mid f_1(\textbf{x}) \le -1\}\]
\begin{center}
\includegraphics[scale = 0.4]{lowerbound.png}
\end{center}
\end{frame}

\begin{frame}
\frametitle{Prove the Lower Bound}

\[\mathcal{Q}'_I = \{\textbf{x}'' \mid \vec{a}_i\cdot \textbf{x}'' \le b_i, i \in [1, m]\}\]
\[\mathcal{P}_i = \mathcal{Q}_I \cap \{\textbf{x}''\mid \vec{a}_i \cdot \textbf{x}'' > b_i\}\]
\[\mathcal{P}_i' = \mathcal{Q}_I \cap \{\textbf{x}''\mid \vec{a}_i \cdot \textbf{x}'' \ge b_i\}\]




Assume (prove by contradiction) $\mathcal{P}_i$ does not lower bound $f_1$.
Let $\textbf{x}''_0\in \mathcal{P}_i$.

\[f_1(\textbf{x}) = \vec{\lambda}\cdot \textbf{x} + \lambda_0 \]
\[\mathcal{P}_i = \mathcal{O} + \mathcal{C}\]

There must be a vector $\textbf{y}''\in \mathcal{C}$ s.t. $\vec{\lambda}\cdot \textbf{y}  < 0$


\end{frame}

\begin{frame}\frametitle{Prove the Lower Bound}

\[\mathcal{Q}' = \mathcal{Q}_I\cap \{\textbf{x}''\mid f_1(\textbf{x}) \le -1\}\]
\[\mathcal{Q}'_I = \{\textbf{x}'' \mid \vec{a}_i\cdot \textbf{x}'' \le b_i, i \in [1, m]\}\]
\[\mathcal{P}_i = \mathcal{Q}_I \cap \{\textbf{x}''\mid \vec{a}_i \cdot \textbf{x}'' > b_i\}\]
\[\mathcal{P}_i' = \mathcal{Q}_I \cap \{\textbf{x}''\mid \vec{a}_i \cdot \textbf{x}'' \ge b_i\}\]

$\textbf{x}_0'' + k\textbf{y}''$ is in $\mathcal{P}_i'$, the set $S = \{\textbf{x}_0'' + k\textbf{y}''\mid k \in \mathbb{Z}_+\}$ is contained in $\mathcal{P}_i$.

Integer points of $\mathcal{P}_i$ are all in $\mathcal{Q}_I\setminus \mathcal{Q}_I'$.

Contradiction.

Hence, $f_1$ is bounded.
\end{frame}

\fi



\fi
\begin{frame}\frametitle{The Depth of a M$\Phi$RF}
Idea: pre-compute the depth $d$ for M$\Phi$RF synthesis.

\begin{theorem}[5]
For integer $B > 0$, the following loop $\mathcal{Q}_B$
\begin{center}
\includegraphics[scale=0.5]{5.PNG}
\end{center}
needs at least $B + 1$ components in any M$\Phi$RF.
\end{theorem}

\begin{proof}
Define $R_I = \{(2^ic, c, 2^{i+1}c, 3c) \mid c \ge 1 \}$ and note that for $i \in [0,B]$, we have $R_i \in \mathcal{Q}_B$. 

Assume the loop has a M$\Phi$RF with depth $B$, then it is obvious that there are $R_i$ and $R_j, i\ne j$ that are ranked by the same phase $f_k$, w.l.o.g., assume $j > i$ and $f_k(x, y) = a_1 x + a_2 y + a_0$, we have 
 
\end{proof}

\end{frame}

\begin{frame}\frametitle{Proof of Theorem (5)}
$j > i$ and $f_k(x, y) = a_1 x + a_2 y + a_0$
\begin{center}
\includegraphics[scale = 0.3]{9.PNG}
\end{center}
$j > i$

$a_12^{i - 1} > 0$ $\Rightarrow$ $a_1 > 0$

$a_1 (2^{i} - 2^{j - 1}) > 0$ $\Rightarrow$ $i + 1 > j$ $\Rightarrow$ $i \ge j$. Contradiction.




\end{frame}

\begin{frame}\frametitle{Iteration Bounds from M$\Phi$RFs}
\begin{example}
\[\texttt{while }(x \ge 0)\texttt{do } x'= x + y, y' = y - 1\]
\end{example}
M$\Phi$RF: $\langle y + 1, x \rangle $

When start from $x = x_0$ and $y = y_0$...

$x_0 + \dfrac{y_0(y_0 + 1)}{2} - 1$
\end{frame}

\begin{frame}\frametitle{Iteration Bounds from M$\Phi$RFs}
Overview:
Given a SLC loop and a corresponding M$\Phi$RF $\tau = \langle f_1, \ldots, f_d\rangle$.
\begin{itemize}
\item $F_k(t)$: the value of $f_k$ after iteration $t$.

\item $UB_k(t)$: bound for $f_k$. For $t > T_k$, $UB_k(T_k)$ becomes negative.

\item $T_k$: an upper bound on the time in which the $k$-th phase ends.

\item The whole loop must terminate before $\max_kT_k$ iterations.
\end{itemize}

$\textbf{x}_t$ be te state after iteration $t$. Define $F_k(t) = f_k(\textbf{x}_t)$. Let $M = \max(f_1(\textbf{x}_0), \ldots, f_k(\textbf{x}_0))$ 

\end{frame}
\begin{frame}\frametitle{Iteration Bounds from M$\Phi$RFs}
\begin{lemma}[4]
For all $k \in [1, d]$, there are $\mu_1, \ldots, \mu_{k - 2} \ge 0$ and $\mu_{k - 1} > 0$ such that $\textbf{x}''\in \mathcal{Q}\rightarrow \mu_1f_1(\textbf{x}) + \cdots + \mu_{k - 1}f_{k-1}(\textbf{x}) + (\Delta f_k(\textbf{x}'') - 1) \ge 0$.
\end{lemma}
\begin{lemma}[5]
For all $k \in [1, d]$, there are constants $c_k, d_k > 0$ such that $F_k(t) \le c_kMt^{k - 1}- d_kt^k$, for all $t \ge 1 $.
\end{lemma}
\end{frame}
\ifcomm
\begin{frame}\frametitle{Iteration Bounds from M$\Phi$RF}
\begin{lemma}[4]
For all $k \in [1, d]$, there are $\mu_1, \ldots, \mu_{k - 2} \ge 0$ and $\mu_{k - 1} > 0$ such that $\textbf{x}''\in \mathcal{Q}\rightarrow \mu_1f_1(\textbf{x}) + \cdots + \mu_{k - 1}f_{k-1}(\textbf{x}) + (\Delta f_k(\textbf{x}'') - 1) \ge 0$.
\end{lemma}
\begin{proof}
\includegraphics[scale=0.35]{13.png}
\end{proof}
\begin{lemma}[5]
For all $k \in [1, d]$, there are constants $c_k, d_k > 0$ such that $F_k(t) \le c_kMt^{k - 1}- d_kt^k$, for all $t \ge 1 $.
\end{lemma}
Proof Idea: Use the bound for $-\Delta f_k(\textbf{x}_i '')$ to bound $F_k(t)$.
\end{frame}

\begin{frame}\frametitle{Proof of Lemma (6)}
\begin{center}
\includegraphics[scale= 0.33]{14.png}
\end{center}
where $ \mu_1f_1(\textbf{x}) + \cdots + \mu_{k - 1}f_{k-1}(\textbf{x}) \ge \Delta f_k(\textbf{x}'') = f_k(\textbf{x}_{i+1} - f_k(\textbf{x}_{i})$
\end{frame}
\fi
\begin{frame}
\begin{theorem}[6]
An SLC loop that has a M$\Phi$RF terminates in a number of iterations bounded by $O(||\textbf{x}_0||_\infty)$
\end{theorem}

\end{frame}
\ifcomm
\begin{frame}
\begin{theorem}[6]
An SLC loop that has a M$\Phi$RF terminates in a number of iterations bounded by $O(||\textbf{x}_0||_\infty)$
\end{theorem}

\begin{proof}
$F_k(t) \le c_kMt^{k - 1}- d_kt^k$. For $t > \max\{1, (c_k/d_k)M\}$, we have $F_k(t) < 0$.

Thus, the loop terminates by the time $\max\{1, (c_i/d_i)M, \ldots, (c_k/d_k)M\}$ where $M = \max(f_1(\textbf{x}_0), \ldots, f_k(\textbf{x}_0))$.

\end{proof}
\end{frame}
\fi

\end{document}